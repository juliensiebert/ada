Émilie est retournée dans la maison. 
Ada s’est assise sur un muret et réfléchit. 
Elle trouve bizarre que l'on ne puisse pas entièrement diviser certains nombres. 
Pendant ce temps, Charles s’amuse sur la ligne des nombres. \\
\guillemotleft Sept! deux! neuf! zéro! huit! six! trois! cinq! un! trois! Regardez, Charles le dompteur de nombres! \\
\mdash Charles, arrête de faire n'importe quoi. Tu ne marches même pas correctement sur la ligne! se plaint Ada. \guillemotright \\
Ne pouvant se concentrer, Ada regarde Charles jouer sur la ligne des nombres. Elle se rend compte qu'il lui faut faire deux pas pour passer du $0$ au $1$, puis encore deux pour passer du $1$ au $2$, etc. Ada réalise que Charles fait des demi-pas (des moitiés de pas d'Ada). 
C’est vrai, se dit Ada, il y a plein de choses qui sont $1$ mais qu’on peut couper en deux : par exemple, une pomme ou un paquet de cartes. Et il y a des choses qui sont $1$ et qu’on ne peut pas couper en deux : par exemple un caillou ou un crayon.
Ada remarque alors un moineau qui sautille dans le jardin. Il fait des pas encore plus petits que ceux de Charles. Peut-être qu'il fait dix pas pour faire un pas d'Ada. Sur le muret où elle est assise, Ada remarque des fourmis. Elles doivent faire au moins cent pas pour faire l‘équivalent d’un pas d'Ada.
Cela lui donne une idée. Si Charles doit faire deux pas quand Ada en fait un, alors un pas de Charles est égal à un pas de Ada coupé en deux : $1/2$. Un pas de moineaux est égal à un pas de Ada coupé en dix : $1/10$. Et un pas de fourmis égale un pas de Ada coupé en cent ! $1/100$.
Ada prend son carnet et écrit $2 \times \mathrm{Charles} = 1 \times \mathrm{Ada}$. Ada imagine deux Charles. Deux petits frères! Oh non, un petit frère c’est déjà beaucoup alors deux, bonjour les dégâts. Ada écrit alors $1 \times \mathrm{Charles} = \mathrm{Ada} / 2$. Ada s’imagine coupée en deux. Cette idée ne la réjouit pas non plus. En plus, elle se demande si elle à le droit de mélanger les nombres avec les lettres.
À ce moment, Charles l’appelle : \\
\guillemotleft Ada ! J’ai attrapé tous les nombres ! J’en veux des nouveaux ! \guillemotright \\
Ada se lève, prend ses craies, et se dirige vers la ligne des nombres.
\guillemotleft Regarde Charles, je vais te dessiner tes nombres à toi. \guillemotright \\
Ada dessine $1/2$ entre le $0$ et le $1$. Ensuite Ada compte. Pour arriver au $1$, Charles doit faire deux pas. Deux demis-pas d’Ada : $2/2 = 1$. Pas besoin de le dessiner celui-là, alors Ada continue. Pour arriver entre le $1$ et le $2$, Charles doit faire trois pas. Ada écrit $3/2$. Pour arriver sur le deux, Charles dois faire quatre pas: $4/2 = 2$. Encore un, qu’Ada n’a pas besoin de dessiner. Entre le $2$ et le $3$, Ada écrit $5/2$, et ainsi de suite. Entre le $3$ et le $4$, $7/2$. Entre le $4$ et le $5$, $9/2$. Ada continue jusqu'à $10$. 
Charles est ravi.\\
\guillemotleft Des nombres rien que pour moi, merci Ada ! Et comment ils s’appellent? \guillemotright \\
Ada réfléchit. Charles fait des demi-pas. \\
\guillemotleft Le premier, $1/2$ , s’appelle un demi. Ensuite, le $3/2$, c’est trois demi. $5/2$, cinq demi, etc. \guillemotright \\
Charles s’applique maintenant à poser un pas sur chacun de ses nombres à lui.
Ada se demande comment on appelle un $1$ coupé en trois ($1/3$), ou un $1$ coupé en quatre ($1/4$). Et les autres? $1/5$, $1/6$, $1/7$, $1/8$, et $1/9$ ? Pour Ada, $1/10$ s’appellera désormais un moineau. Et $1/100$, une fourmi. 
Ces nouveaux nombres sont étrange, pense Ada, quand on coupe $1$ par un plus grand nombre, le résultat devient plus petit.
$1/100$ est plus petit que $1/10$. $1/1000$ est plus petit que $1/100$. $1/100000$ est encore plus petit ! Ada s’imagine partir à la chasse du plus petit nombre et qu'elle devient minuscule, plus petite qu’un moineau, plus petite qu'une fourmi, infiniment petite.
