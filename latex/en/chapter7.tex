Emily returned to the house. 
Ada sits on a low wall and thinks. 
She thinks it's strange that some numbers can't be completely divided. 
Meanwhile, Charles is having fun on the number line. \\
"Seven! two! nine! zero! eight! six! three! five! one! three! Look, Charles the number tamer! \\
"Charles, stop messing around. You're not even walking properly on the line!" complained Ada." \\
Unable to concentrate, Ada watches Charles play on the number line. She realizes that it takes him two steps to go from $0$ to $1$, then two more to go from $1$ to $2$, and so on. Ada realizes that Charles is taking half-steps. 
It's true, Ada says to herself, there are lots of things that are $1$ but can be cut in half: for example, an apple or a deck of cards. And then there are things that are $1$ and can't be cut in half: a pebble or a pencil, for example.
Ada notices a sparrow hopping around in the garden. Its steps are even smaller than Charles'. Maybe it takes ten steps to take one of Ada's. On the low wall where she's sitting, Ada notices some ants. They must take at least a hundred steps to take the equivalent of one of Ada's. This gives her an idea.
If Charles has to take two steps when Ada takes one, then one step from Charles is equal to one step from Ada cut in half: $1/2$. One sparrow step is equal to one Ada step cut in ten: $1/10$. And one step of ants equals one step of Ada cut into a hundred! $1/100$.
Ada takes out her notebook and writes $2 \times \mathrm{Charles} = 1 \times \mathrm{Ada}$. Ada imagines two Charles. Two little brothers! Oh no, one little brother is already a lot, so two is a bit much. Ada writes $1 \times \mathrm{Charles} = \mathrm{Ada} / 2$. Ada imagines herself cut in two. She's not happy about the idea either. What's more, she wonders if she's allowed to mix numbers with letters.
Just then, Charles calls out to her.
"Ada! I've got all the numbers! I want some new ones!" \\
Ada gets up, grabs her chalk and heads for the number line.\\
"Look, Charles, I'm going to draw your numbers for you." \\
Ada draws $1/2$ between the $0$ and the $1$. Then Ada counts. To get to $1$, Charles has to take two steps. Ada's two half-steps: $2/2 = 1$. No need to draw this one, so Ada continues. To get between $1$ and $2$, Charles has to take three steps. Ada writes $3/2$. To get to the two, Charles has to take four steps: $4/2 = 2$. One more, which Ada doesn't need to draw. Between the $2$ and the $3$, Ada writes $5/2$ and so on. Between $3$ and $4$, $7/2$. Between the $4$ and the $5$, $9/2$. Ada continues to $10$. 
Charles is delighted.\\
"Numbers just for me, thanks Ada! And what are their names? " \\
Ada thinks. Charles takes half-steps. \\
"The first, $1/2$, is called one half. Next, $3/2$, is three halves. $5/2$, five halves, and so on." \\
Charles now takes a step on each of his numbers.
Ada wonders what $1$ cut into three ($1/3$) or $1$ cut into four ($1/4$) is called. What about the others? $1/5$, $1/6$, $1/7$, $1/8$ and $1/9$? For Ada, $1/10$ will now be called a sparrow. And $1/100$, an ant. 
These new numbers are strange, Ada thinks. When you cut $1$ by a larger number, the result becomes smaller.
$1/100$ is smaller than $1/10$. $1/1000$ is smaller than $1/100$. $1/100000$ is even smaller! Ada imagines herself on the hunt for the smallest number and becoming tiny, smaller than a sparrow, smaller than an ant, infinitely small.
