Ada continued to play with additions and subtractions. 
Emily left her notebook and pencil and went to pick flowers in the garden. 
After a few rounds, Ada has a problem. 
She ran out of digits! 
She wrote $4 + 5 + 2 - 3$ in the notebook, stood on the $4$, took five steps forward and now she is stuck. 
She is now on $9$, the last digit in the line. 
Ada has to take two steps forward and then three steps back.
But after $9$, Ada hasn't drawn anything.
She could walk forward and then backward along the line without any digits to put her feet on. But how does she know if she's in the right place? 
She knows that there are things after $9$. 
She has heard of numbers like twenty-one, sixty-three, thirty-six and fifteen. 
But Ada doesn't know how to draw them, or even in what order.
Which comes first? twenty-one or fifteen? sixty-three or thirty-six?
She was quite worried. What should she do? Ask her aunt Emily? No, Emily was probably very busy, and Ada thought she could do it on her own. So she thought about it. She thought she could come up with some new digits. 
All you have to do is draw new symbols and find new names for them. E
asier said than done. After four new digits, "ga", "bu", "zo" and "meu"\footnote{Author's note: these numbers were not invented by Ada, but come from a cartoon series called The Shadoks, see \url{https://en.wikipedia.org/wiki/Les_Shadoks}.}, Ada has run out of ideas.
It's not that easy. You have to think of new names and new symbols to draw (and who invented the digits $0$ to $9$? Ada wonders). 
And it's hard to remember things like that: she was already having trouble remembering the names of the digits $0$ to $9$, so imagine if she had to remember a new name and symbol for each new digit. 
So, no, Ada decided that inventing new digits might not be such a good idea after all. But what could she do? 
Still standing on the $9$, Ada had to take two more steps forward and three backwards. 
She's thinking again. Maybe she could use the digits again? 
After $9$, you can write $0$, $1$, $2$, etc. again. That way you don't have to learn new symbols. That's good, thinks Ada, but the problem is that you can get lost.
How will I know how far away I am from the house? If I'm on a $5$, am I on the first $5$ - the one next to the house, the second? the third? How can I find my way around, Ada wondered.
She thought again. What if she used colours? 
One colour for the first series of digits from $0$ to $9$, say green. Then another colour, say yellow, for the second series of digits, then red, then blue. Hmm, is blue bigger than red? Or would it be better to use blue for the third series and red for the fourth? And what colour for the fifth? Ada says to herself that this solution causes more problems than it solves.  
The colours may not be the right solution, but it gives her another idea. Ada is going to write down the number of times she uses all the digits. She starts at home: $0$, $1$, $2$, $3$, $4$, $5$, $6$, $7$, $8$ and $9$. After $9$, Ada starts the sequence again. She writes down $0$. And so that she doesn't forget that she just used the whole series the first time, she writes $1$ to the left.
Ada then gets $1~0$ ('one-zero'). She goes on to write $1~1$ ('one-one'), $1~2$ ('one-two'), $1~3$ ('one-three'), $1~4$, $1~5$, $1~6$, $1~7$, $1~8$ and $1~9$. At this point, Ada extends the line along the path to the garden. Since she has used the whole series of numbers a second time, she will now write $2~0$ ('two-zero') and continue: $2~1$ ('two-one'), $2~2$ ('two-two'), $2~3$, $2~4$, $2~5$. Ada pauses for a moment to look at her solution. She likes it. First of all, you never get lost, she thinks (you always know whether you're near or far from home), and then you can draw in any colour you like! Emily, who had just returned from the garden, said to her:\\
"What are you doing on twenty-five?"\\
Ada asks, "Is 'two-five' actually called twenty-five?"\\
"Yes," answers Emily, "all numbers have names."\\ 
"Numbers?" asks Ada.\\
"Yes," says Emily, "that's what we call them. The digits are the symbols: $0$, $1$, $2$, $3$, $4$, $5$, $6$, $7$, $8$ and $9$. There are only ten of them. You can write numbers with digits. For example, twenty-five with a two and a five: 25."\\
"And what's the one with a one and a four?" asks Ada.\\
"It's called fourteen."\\
"And what about this one? The funny one with a one and a seven?"\\
"It's seventeen."\\
Emily then taught Ada the names of the numbers: "ten" (10), "eleven" (11), "twelve" (12), etc. up to 25, where Ada stopped. Ada wonders who decides the names of the numbers. Emily explains that the digits have travelled a long way: China, India, the Middle East, Central Asia, North Africa and Europe. Numbers as we know them are used all over the world. Ada finds this very practical. She imagines a world where numbers change from one country to another. How complicated!