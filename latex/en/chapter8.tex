Émile is now back. Unfortunately without any candy. It's a shame, Ada thinks, but it doesn't matter, Ada has to show Emily things first. We'll deal with the candy later.\\
"Look, Emily," says Ada, "Charles takes half-steps. When I take one step, he has to take two."\\
"Is that right? And how much do Emily's steps equal Ada's steps," asked her aunt.\\
"I don't know, but let's give it a try," replies Ada. "I've got an idea. We stand next to each other, walk together and see how many steps we have to take to get to the same place."\\
Ada and Emily both sit in front of the house, on the $0$. Ada takes a step and arrives on the $1$. Emily also takes a step, but her steps are bigger and she's stuck between the $1$ and the $2$.
Ada takes a second step and passes Emily. Emily takes a second step and arrives on the $3$. Ada has to take a third step to reach Emily.
Ada writes $3$ Ada $= 2$ Emily in her notebook. Three steps from Ada are worth two steps from Emily. Sounds complicated, says Ada. I'd rather know how much one of my steps is worth in Émile steps. If three of my steps equal two of Emily's steps, then by cutting into three, I get one of Ada's steps equal to two of Emily's steps cut into three. Ada notes $1$ Ada $= 2/3$ Emily.
Emily looks at the notebook and exclaims\\
"Oh, the beautiful equations!"\\
"Equations," says Ada, astonished.
"Equations," resumes Emily. "They're the formulas you just wrote down. That's what we call them. They're like recipes that tell us how to transform things."\\
I knew it," says Ada, "you can mix numbers and letters."\\
"Yes, as long as it adds up!"\\
"I wonder how many steps you'd have to take if I got to number 12," said Ada.\\
"Well, we could try," replies Emily. I stand on the $0$. I take one or two steps forward and I'm on the $3$. After three and four steps, I'm on $6$. Five and six steps to $9$. Seven and eight steps. And that's it! I'm on $12$. Eight steps, that's your answer Ada."\\
"Hey, don't leave me alone!" suddenly shouts Charles from the house.\\
"You're not alone," replies Ada, "and what's more, you're not even far away."\\
"If I'm very far, at least a hundred and fifty-eight!"\\
"Nonsense," says Ada, "we're on $12$ and you're on $0$. Since you're doing half-steps, you must be doing twice as many as me. That's two packs of 12 and that makes...," Ada calculates in her head, "twenty-four, right Emily?"\\
"Yes, that's right. Come on, Charles, it's not that far."\\
Charles runs to join them. \\
"And now," says Emily, "I have to go get some salad from the garden. Who's coming with me?"\\
"Me!" exclaimed Charles.\\
"Are you joining, Ada?"\\
"I don't know, the salads are at the bottom of the garden and that seems a long way away. What would be nice," continues Ada, "is to have a method of knowing how many steps I have to take to reach you. The best thing would be to have the answer right away, just by looking, without calculating."\\
"I can draw you the equations," suggests Emily.\\
"But I've already written them in my notebook!"\\
"No, look."\\
Emily draws two lines in the notebook: a horizontal one for her steps and a vertical one for Ada's steps. Emily then notes $0$ where the lines cross, and adds numbers from $1$ to $9$ on each. She shows Ada how to draw the equation. When Emily takes two steps, Ada must take three. Emily places the pen on the $2$ number on the horizontal line and moves up to the $3$ number on the vertical line. There, she scores her first point. When Emily takes four steps, Ada must take six. In the same way, Emily scores a second point. When Emily takes six steps, Ada must take nine. Finally, Emily marks a third point on the sheet. Then she draws a line connecting all the dots.\\
"You see," says Emily, "this line I've just drawn represents my equation. If you want to know how many steps you have to take to reach me, just follow this line."\\
"Follow the line?"\\
"Yes, look. If I take eight steps, the line tells me you have to take twelve."\\
"I already knew that," replies Ada.\\
"If I take ten steps, you have to take fifteen."\\
"And if you only do five?" asks Ada, "what does the line tell us?"\\
"It tells us you have to do seven and a half."\\
Ada looks at the drawing in her notebook. After magic spells and recipes, here's a treasure map.
