Ada picks up where she left off. 
She takes the notebook and reads $4 + 5 + 2 - 3$.
Ada places herself on the $4$ and takes five steps forward. 
She is now on $9$. 
She takes two steps forward. She is now on $11$. 
Ada turns around and takes three steps in the opposite direction, stopping on $8$.
At last! Ada can now write $4 + 5 + 2 - 3 = 8$ in her notebook.
She turns to Emily and asks her: \\
"Say, Auntie, what's the biggest number? \\
Emily looked at her and smiled.\\
"The biggest number you've drawn on your line so far is twenty-five."\\
"No!" replied Ada, "you know, the biggest of ALL numbers. After 25 I can write $2~6$ (Ada then pronounces \qq{two-six} and Emily tells her its name: \qq{twenty-six}), and $2~7$ (\qq{two-seven}, \qq{twenty-seven}), $28$, $29$. Then I write $30$ to say that I've used all the digits from $0$ to $9$ three times. And I continue: I write $31$, $32$, $33$ and so on. You see, if the line continues through the garden, to the neighbour's house, across the road, I should be able to write $99$. If there's still room on the line (and if Mum lets me go to the neighbours, cross the road and continue on the line), I should be able to go one step further and add $1$, right?"\\
"That's right," answers Emily, "you can always go one step further, or add $1$. After $99$ comes $100$, after $999$ comes $1000$, after $9999$ comes $10,000$."\\
"So," asks Ada, "which is the bigger number? Because if you can always add $1$, that means there's always a bigger number, and another bigger number than the biggest of the biggest. It never ends!"\\ 
Ada was almost dizzy. \\
"It's true," explains Emily, "you're right. It never ends. We say there are infinite numbers and we write $\infty$."\\
"Like an eight lying down?"\\
"Yes, like the digit eight. But be careful, infinity is not a number. As you said, you can always find a greater number, and another greater number, and another greater number than the greatest of the greatest, without ever stopping. That's what we mean when we write $\infty$."\\
Ada watches this newcomer. When she looks at him, she sees a path that never ends. She thinks she could travel very far in the land of numbers.

