Ada's little brother Charles joins his sister and Emily for a snack. After eating a few cakes, Ada asked Emily:\\
"Say, Auntie, can you jump with both feet? Look at that!"\\
Ada stands on $0$, makes her first jump and lands on $2$. A second jump and she's on $4$. Another and Ada lands on $6$.
"Look, Emily, every time I jump, I go two steps further! Now it's your turn!"\\
Emily also places herself on the $0$, makes her first jump and lands on the $3$. One more and she's on $6$ too.\\
"My turn!" shouts Charles.\\
And he too starts jumping. From the $0$, he lands on the $1$. Then onto the $2$ and the $3$, and so on, until, after six jumps, he arrives with Emily and Ada on the $6$.
Ada then wonders where she'd land if she did 10 jumps in a row. 
She starts again from the beginning ($0$). After three jumps, she's back on $6$. 
She makes a fourth jump and lands on $8$. 
After a fifth, she's on $10$. She continues counting the number of jumps and arrives, after ten, on $20$.\\
It's your turn now," says Ada to Emily.\\
Emily stands on $0$ and makes one jump (landing on $3$), then a second (landing on $6$) and a third (on $9$), and so on. After the tenth jump, she's much further away than Ada, at number $30$.
Ada is amazed; she and Emily have each made ten jumps. But since Emily's jumps are bigger than Ada's, Emily has gone much further than Ada. 
And now it's your turn, Charles," says Ada to her brother."\\
And Charles starts jumping ten times in a row. Starting at $0$, he finally reaches $10$.
Ada wants to remember all these numbers. She takes out her notebook and writes:
\begin{description}[leftmargin=0.5cm]
    \item $2+2+2+2+2+2+2+2+2+2=20$ for her,
    \item $3+3+3+3+3+3+3+3+3+3=30$ for Emily and
    \item $1+1+1+1+1+1+1+1+1+1=10$ for Charles.
\end{description}
How long and complicated, Ada thinks. 
Imagine, if we had to make 50 jumps or 100, there wouldn't be room to write it all down! 
We'd need a faster way of writing that we'd done the same thing 10 times. We'd only need two numbers: for example, the number of jumps ($10$) and the size of each jump ($2$). It would be a lot more practical. 
Ada shows Emily her notebook and asks if there's a simpler way of writing than $2+2+2+2+2+2+2+2+2=20$. Emily then introduces him to a new symbol: a little cross called \qq{fois} that goes like this: $\times$. Emily shows her that $2+2+2+2+2+2+2+2+2=20$ can be written $10 \times 2 = 20$ and that we say ten times two.
Ada looks at this newcomer and then writes $10 \times 3 = 30$ in her notebook. 
\\
It's still more practical than writing $3+3+3+3+3+3+3+3+3=30$," she says, "and that's exactly what I needed. And for you, Charles, we write $10 \times 1 = 10$."\\
But something is troubling Ada.\\
"Say, Emily," she asks, "you told me that when you write $+$, you're adding. When we write $-$, we subtract. What do we do when we write $\times$?"\\
"It's called multiplication," answers Emily, "we say we multiply two numbers together."\\
"So if I did $50$ jumps, what number would I arrive at?"\\
"A hundred, because $50 times 2 = 100$."\\
"And if I do $100$?"\\
"Two hundred, because $100 \times 2 = $200."\\
Ada imagines she'd be very tired from jumping so much.
