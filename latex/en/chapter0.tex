This is Ada. Today she decided to go outside and play with her chalks. 
She wants to draw the digits she has just learnt at school. They have funny shapes and strange names.
First Ada draws a line. It starts at the house and goes across the yard to the garden. 
It will be a good place to put the digits. They mustn't fall off! 
Ada starts by drawing a circle near her house. This is the $0$. It will be the starting point. 
Then she takes a step towards the garden and follows the line. Ada now draws a $1$. It's easy, it looks like a vertical bar. 
A second step towards the garden and she draws a $2$. This one is a bit more complicated: it turns, goes up and down. And don't forget the little horizontal bar underneath.
One more step and Ada draws a $3$. Two rounded lines and you're done.
Next come the $4$ (all straight lines), the $5$ (a mixture of straight and rounded lines), the $6$ (all rounded lines), the $7$ (like an unfinished zigzag), the $8$ (like two circles) and the last one: the $9$ (be careful not to confuse it with a $6$).  
Ada is now in the middle of the yard. She takes a step back and goes from $9$ to $8$. She takes another step back and there she is at $7$, then $6$, $5$, $4$, $3$, $2$, $1$ and $0$, right in front of the house. 
It's funny, she says to herself, when I take a step towards the garden, I step on the bigger number. And when I step backwards, I step on the smaller number. 
Let's see what happens when I take three steps forward.
Ada stands on the $0$ and counts as she walks: $1$, $2$ and $3$! Three more steps: $4$, $5$ and $6$! And one more: $7$, $8$ and $9$! \\
"And now," says Ada, "in the other direction! Let's see what happens when I go back two steps: 
\begin{description}
    \item $8$ and $7$ ! 
    \item $6$ and $5$ ! 
    \item $4$ and $3$ ! 
    \item $2$ and $1$ !"
\end{description} 
I'm almost there, Ada thinks.