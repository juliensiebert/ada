Ada is still a little hungry.
There's no more cake, but fortunately Emily has brought some raspberries from the garden.
There are twelve of them. Ada wonders how to share them. 
Emily and Charles also like raspberries. 
Ada thinks. I'll start by giving one to Charles, one to Emily and one to me. 
There are (Ada counts the raspberries aloud) nine left. 
Here's a second one for Charles, a second one for Emily and a second one for me. Only six left. 
So here's a third for Charles, Emily and me. There are now only three left. 
Finally, Ada, give each of you a fourth raspberry. And that's the twelve raspberries shared.\\
"Who's still hungry? Who wants raspberries?" asks Ada. "Look, we all have four."\\
"Bravo," says Emily, "you know the division too!"\\
"The what?" asks Ada.\\
"Division," explains Emily. "It is what you just did to share the raspberries. Division is a friend of multiplication, subtraction and multiplication."\\
"Ada replies, her mouth full of fruit. And how do you draw it?"\\
"We often use a line to show that we're dividing (cutting) one number by another."\\
"We cut the numbers!" exclaims Ada. "\\
Ada imagines numbers cut in half. An $8$ having lost its upper half (\clipbox{0 -0.5 0 {0.4\totalheight}}{$8$}), a footless $4$ (\clipbox{0 {0.3\totalheight} 0 0}{$4$})! Strange!.\\
"And it doesn't hurt?"\\
"No," reassures Emily. "Here, take a look."\\
Emily takes the notebook and writes: $12/3=4$.\\
"Twelve raspberries divided into three equals four raspberries per person. You can also write like this."\\
Emily hands Ada the notebook and shows her: 
$$\frac{12}{3}=4$$
Ada thinks the division is rather pretty. She wonders if it works for all numbers and not just raspberries. 
She'd like to know what $7$ divided by $3$ is. How do you do it? For the raspberries, it was easy: all you had to do was hand them out. One for Ada, one for Charles, one for Emily, two for Ada, two for Charles, two for Emily, and so on, until there were no raspberries left. 
But Ada's run out of raspberries to try. 
Ada thinks. She has an idea. She's going to use her coloured chalks and some pebbles.
Ada gets up and takes three pieces of chalk: one green, one orange and one purple. Then she draws three circles on the ground: one green, one orange and one purple. Then she picks up seven pebbles. First she puts one in the green circle, one in the orange circle and one in the purple circle. It's the same as with the raspberries, except that Ada distributes the pebbles. Ada continues. She has four pebbles left. She puts one in the green circle, one in the orange circle and one in the purple circle. Each circle now has two pebbles. Ada looks in her hand. There is only one pebble left to distribute. Not enough to put one in each coloured circle! It's not going to work.\\
"Emily? Your division isn't working.""\\
"Really?" asks Emily. \\
"Yes," resumes Ada, "I wanted to divide $7$ by $3$. I have three baskets: one green, one orange and one purple. I already put two pebbles in each basket. But I've still got one more to go, but I can't divide it into three!"\\
"It's normal," explains Emily, "not all numbers can be divided completely. Sometimes there's a remainder." \\
"A remainder?" \\
"Yes, a remainder, that's what it's called. In your case, $7$ divided by $3$ equals $2$, leaving $1$. You could also say that $7$ equals $3$ packets of $2$ plus $1$. All the same. And we can write it down like this: $7=3\times2+1$." \\
"Huh?" exclaims Ada. "What happened to my division? You're cheating again, Auntie!" \\
"Of course not. Multiplication and division are such good friends that you can change one into the other. We say that division is the inverse of multiplication."\\
Ada looks at her stunned. Then after a while she says: \\
"Okay, so please, can you get us six candies? Two each? Because $6 = 3 = $2 and $6 / 3 = $2, right?" 

