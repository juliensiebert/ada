Ada feels proud: she has figured out how to write numbers bigger than $9$, and she has also discovered an infinity of numbers. 
But Ada is also a little sad, because in her playground she can only play with a few numbers. And what's more, they're small: most of them only have two digits!
Seeing this, Emily asks her if she still wants to play with additions and subtractions. 
But Ada is no longer amused. She feels there is nothing left to discover. 
Emily suggests going for a snack. But before leaving, she writes down a new formula in the notebook: $3 - 5 = ~$ and hands it to Ada. \\
"Such a small formula," thinks Ada, "can't be too complicated. 
I'll solve it (as Emily says) before Émily comes back."\\ 
Ada stands on the $3$ and starts to move backwards: one step, two steps, three steps and... 
Oh no, here it comes again! Ada is now on the $0$, in front of the house door, at the beginning of the line. 
She still has two steps to go to finish her formula, but the line stops. 
So Ada picks up her chalk and continues the line behind the house to the garage. 
She can see that she needs to go back two steps to finish her formula. 
There is enough room for at least twenty more steps, Ada thinks. 
The problem is, there is no number on this side of the line. 

Ada thinks. 
She could reuse the numbers she already knows. Starting from in front of the house ($0$), she takes one step back and notes $1$, then two steps back and notes $2$. 
The problem, Ada says to herself, is that we can't tell the difference between the $1$ behind the house and the $1$ in front, in the yard. 
She thought she could use some color. 
Green for the front and orange for the back. 
Color is fine, but Ada finds another problem. 
When she goes backwards, i.e. when she removes one, the numbers get bigger and bigger. And when she moves forward, i.e. adds one, the numbers get smaller and smaller.
It is the opposite of before! You can't have rules for one side of the house and other rules for the other side. \\
"We'll end up mixing everything up," says Ada.\\
She thinks again, looks at her notebook with all its formulas and thinks: when it says $-1$, that means I take one off and go back one step. When it says $-2$, it means I take off two and go back two steps. When it says $-3$, it means I'm taking off three and going back three steps. 
Ada stands in front of the house again, on the $0$. She takes a step back and writes down $-1$ (I've taken a step back from zero). She takes another step back and notes $-2$ (I've gone back two steps from zero). She continues to list the steps, writing $-3$, $-4$, $-5$, etc. all the way to the garage, where Ada writes down $-21$. 
Just then, Emily returns with snacks.\\
"Ada, you've discovered negative numbers!" exclaims her aunt.\\
Ada is proud of herself. Emily explains to her that negative numbers have also traveled a long way and have long been considered strange numbers. Ada says to herself that there is nothing weird about them and now, thanks to them, she can finish Emily's formula. Ada places herself on the $3$ and takes five steps back. Now she stands on $-2$.\\
"There," says Ada, "$3 - 5 = 2$. And now, can we get a snack?"

