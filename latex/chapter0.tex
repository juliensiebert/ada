Voici Ada. Aujourd'hui elle a décidé d'aller jouer dehors avec ses craies. 
Elle a envie de dessiner les chiffres qu'elle vient d'apprendre à l'école. Ils sont rigolos avec leurs formes et leurs noms bizarres.
Tout d'abord, Ada commence par tracer une ligne. Elle part de la maison et traverse la cour, jusqu'au jardin. 
Ce sera bien pour poser les chiffres dessus. Il ne faudrait pas qu'ils tombent ! 
Ada commence par dessiner un rond près de sa maison. C'est le $0$. Ce sera le point de départ. 
Ensuite, elle fait un pas vers le jardin en suivant la ligne. Ada dessine maintenant un $1$. C'est facile, il ressemble à une barre verticale. 
Un deuxième pas vers le jardin et elle dessine un $2$. Celui-là est un peu plus compliqué : ça tourne, ça monte et ça descend. Et bien sûr, il ne faut pas oublier la petite barre horizontale du dessous.
Encore un pas et Ada dessine un $3$. Deux arrondis et c'est fini.
Ensuite viennent le $4$ (tout en lignes droites), le $5$ (un mélange de droites et d'arrondi), le $6$ (tout en arrondi), le $7$ (comme un zigzag pas fini), le $8$ (comme deux ronds) et le dernier : le $9$ (attention à ne pas le confondre avec un $6$). 
%\begin{figure}[!h]
%    \centering
%    \vspace{8cm}
%    \caption{Dessine Ada dans son jardin avec tous ses chiffres.}
%    \label{fig:my_label}
%\end{figure}
Ada est maintenant au milieu de la cour. Elle recule d'un pas et passe du $9$ au $8$. Elle recule encore et la voici au $7$, puis au $6$, au $5$, au $4$, au $3$, au $2$, au $1$, et au $0$, juste devant la porte de la maison. 
C'est rigolo, se dit Ada, quand j'avance d'un pas vers le jardin, je marche sur le chiffre plus grand. Et quand je recule, je marche sur le chiffre plus petit. 
Voyons ce qui se passe si j'avance de trois pas. Ada se place sur le $0$ et compte en marchant : $1$, $2$, et $3$ ! Encore trois pas : $4$, $5$, et $6$ ! Et encore une fois : $7$, $8$, et $9$ ! \\
« Et maintenant, dit Ada, dans l'autre sens ! Voyons ce qui se passe si je recule de deux pas : 
\begin{description}
    \item $8$, et $7$ ! 
    \item $6$ et $5$ ! 
    \item $4$ et $3$ ! 
    \item $2$ et $1$ ! »
\end{description} 
Me voici presque arrivée, pense Ada.


%\begin{figure}[!h]
%    \centering
%    \begin{tikzpicture}[thick]
%    
\fill (0,0) circle[radius=2pt];
\node at (-0.25,-0.3) {1};
\fill (1,0) circle[radius=2pt];
\node at (1.25,-0.3) {2};
\fill (1,2) circle[radius=2pt];
\node at (1.25,1.7) {3};
\fill (2,2) circle[radius=2pt];
\node at (2.25,1.7) {4};
\fill (2,3) circle[radius=2pt];
\node at (2.25,3.3) {5};
\fill (1,3) circle[radius=2pt];
\node at (1.25,3.3) {6};
\fill (1,6) circle[radius=2pt];
\node at (1.25,6.3) {7};
\fill (0,6) circle[radius=2pt];
\node at (-0.25,6.3) {8};
\fill (-2,3) circle[radius=2pt];
\node at (-2.25,3.3) {9};
\fill (-2,2) circle[radius=2pt];
\node at (-2.25,1.7) {10};
\fill (0,2) circle[radius=2pt];
\node at (-0.25,1.7) {11};
\draw (0,3) -- (0,3.75) -- (-0.5,3) -- (0,3);
\draw (0,0.1) -- (0,1.9);
%    \end{tikzpicture}
%    \caption{Relie les points pour trouver le chiffre préféré d'Ada}
%    \label{fig:}
%\end{figure}