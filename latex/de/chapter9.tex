Während Emilie Salat pflückt, spielen Ada und Charles zusammen \qq{Ada sagt}. Das Ziel ist einfach: Ada gibt Charles Befehle, die dieser nur ausführen darf, wenn Ada die Formel \qq{Ada sagt} ausspricht.\\
\frqq{}Ada sagt, geh drei Schritte zurück!\flqq{}\\
Charles tritt tatsächlich drei Schritte zurück.\\
\frqq{}Bravo! Ada sagt, hüpfe!\flqq{}\\
Charles stellt sich auf einen Fuß und beginnt zu springen.\\
\frqq{}Jetzt laufe fünf Schritte vorwärts!\flqq{}\\
Er hält es nicht mehr aus und rennt zu Ada.\\
\frqq{}Verloren! Ich habe nicht gesagt "Ada sagt".\flqq{}\\
\frqq{}Oh nein! Jetzt bin ich dran! Charles sagt $3 \cdot (-2)$!\flqq{}\\
\frqq{}$3 \cdot (-2)$!?\flqq{}, wundert sich Ada. \frqq{}Moment mal...\flqq{}, Ada denkt nach und sagt: \frqq{}$-2$ bedeutet, dass wir zwei Schritte zurückgehen, also bedeutet $3 \cdot (-2)$, dass wir dreimal zwei Schritte zurückgehen.\flqq{}\\
Ada geht sechs Schritte zurück.\\
\frqq{}Charles sagt $(-2) \cdot 3$!\flqq{}\\
Ada geht wieder sechs Schritte zurück.\\
\frqq{}Hey, du schummelst!\flqq{}, ruft Charles.\\
\frqq{}Was soll das heißen, ich schummle?\flqq{}, erwiderte Ada.\\
\frqq{}Ja, das stimmt! Früher war es $3 \cdot (-2)$ und jetzt ist es $(-2) \cdot 3$, das ist nicht das Gleiche, also kann es nicht dasselbe sein.\flqq{}\\
\frqq{}$(-2) \cdot 3$ ist, zwei Schritte zurückzugehen dreimal, ist wie dreimal zwei Schritte zurückzugehen. $3 \cdot (-2) = (-2) \cdot 3$.\flqq{}\\
In diesem Moment kommt Emilie mit einem Salat in jeder Hand.\\
\frqq{}Ada hat recht. Die Multiplikation ist kommutativ.\flqq{}\\
\frqq{}Kommuta-was?", fragten die Geschwister im Chor.\flqq{}\\
\frqq{}Es ist kommutativ. Das bedeutet, dass wir die Zahlen um das Zeichen $cdot$ herum umstellen können: $5 \cdot 4 = 4 \cdot 5$, $7 \cdot 3 = 3 \cdot 7$, oder $3 \cdot (-2) = (-2) \cdot 3$. Das funktioniert auch bei der Addition: $1 +3 = 3 + 1$, $5 + 2 = 2 + 5$.\flqq{}\\
\frqq{}Was ist mit Subtraktion und Division?\flqq{}, fragt Ada.\\
\frqq{}Sie, sie sind nicht kommutativ. Man kann die Zahlen nicht beliebig verschieben: $3 - 2$ ist nicht gleich $2 - 3$ und $6/3$ ist nicht gleich $3/6$.\flqq{}\\
Ada schaut Charles an.\\
\frqq{}Das heißt, ich habe gewonnen!\flqq{}\\
\frqq{}Nein, noch einer! Charles sagt: $(-4) \cdot (-3)$!\flqq{}\\
Ada denkt nach. Sie hat noch nie zwei negative Zahlen multipliziert.
\frqq{} Wenn man zwei Schritte vorwärts geht, schreibt man $2$ und wenn man zwei Schritte rückwärts geht, schreibt man $(-2)$. Wenn wir $4 \cdot (-3)$, vier Packungen $(-3)$, hätten, würden wir viermal drei Schritte zurückgehen.\flqq{}\\
Ada rechnet nach.\\
\frqq{}Wir würden 12 Schritte zurückgehen. Und da wir $ (-4) \cdot (-3)$ haben, machen wir das Gleiche in die andere Richtung. Also gehen wir 12 Schritte vorwärts.\flqq{}\\
Ada fragt Emilie.\\
\frqq{}Sag mal, Tante, ist $(-4) \cdot (-3)$ gleich $12$?\\
\frqq{}Bravo Ada\flqq{}, antwortet Emilie, \frqq{}wenn man mit positiven und negativen Zahlen multipliziert, muss man mit den Plus- und Minuszeichen jonglieren.\flqq{}\\
\frqq{}Mit Zeichen jonglieren? Das klingt schwierig.\flqq{}\\
\frqq{}Nein, das hast du gerade getan. Zwei positive Zahlen miteinander oder zwei negative Zahlen miteinander zu multiplizieren, ergibt immer ein positives Ergebnis. Die Multiplikation einer positiven Zahl mit einer negativen Zahl führt zu einem negativen Ergebnis.\flqq{}\\
Plötzlich ruft eine Stimme aus dem Haus nach den Kindern:\\
\frqq{}Ada, Charles! Es ist Zeit, nach Hause zu gehen!\flqq{}\\
\frqq{}Wir kommen!", antworteten die Geschwister einstimmig.\flqq{}