Emilie ist wieder ins Haus gegangen. 
Ada hat sich auf eine kleine Mauer gesetzt und denkt nach. 
Sie findet es seltsam, dass man manche Zahlen nicht vollständig teilen kann. 
In der Zwischenzeit spielt Charles auf der Zahlenlinie. \\
\frqq{}Sieben! zwei! neun! null! acht! sechs! drei! fünf! eins! drei! Seht her, Charles, der Zahlendompteur!\flqq{}\\
\frqq{}Charles, hör auf mit dem Quatsch. Du kannst ja nicht mal richtig auf der Linie laufen\flqq{}, beschwert sich Ada.\\
Da Ada sich nicht konzentrieren kann, schaut sie Charles zu, wie er auf der Zahlenlinie spielt. Sie bemerkt, dass er zwei Schritte machen muss, um von $0$ auf $1$ zu kommen, dann wieder zwei Schritte, um von $1$ auf $2$ zu kommen usw. Ada erkennt, dass Charles halbe Schritte macht (die Hälfte von Adas Schritten). 
Stimmt, denkt Ada, es gibt viele Dinge, die $1$ sind, die man aber halbieren kann: zum Beispiel einen Apfel oder einen Stapel Karten. Und es gibt viele Dinge, die $1$ sind, die man aber nicht halbieren kann: zum Beispiel einen Stein oder einen Bleistift.
Dann bemerkt Ada einen Spatz, der im Garten herumhüpft. Er macht kleinere Schritte als Charles. Vielleicht macht er zehn Schritte für einen Schritt von Ada. Auf dem Mäuerchen, auf dem sie sitzt, bemerkt Ada Ameisen. Sie müssen mindestens hundert Schritte machen, um Adas Schrittlänge zu erreichen.
Das bringt sie auf eine Idee. Wenn Charles zwei Schritte gehen muss, wenn Ada einen geht, dann ist ein Schritt von Charles gleich einem Schritt von Ada, geteilt durch zwei: $1/2$. Ein Schritt der Spatzen ist gleich einem Schritt von Ada geteilt durch zehn: $1/10$. Und ein Schritt der Ameisen entspricht einem Schritt von Ada geteilt durch hundert: $1/100$!
Ada nimmt ihr Notizbuch und schreibt $2 \cdot \mathrm{Charles} = 1 \cdot \mathrm{Ada}$. Ada stellt sich zwei Charles vor. Zwei kleine Brüder! Oh nein, ein kleiner Bruder ist schon zu viel, aber zwei sind noch schlimmer. Ada schreibt $1 \cdot \mathrm{Charles} = \mathrm{Ada} / 2$. Ada stellt sich vor, wie sie in zwei Hälften geteilt wird. Auch diese Vorstellung macht sie nicht glücklich. Außerdem fragt sie sich, ob sie die Zahlen mit den Buchstaben vermischen darf.
In diesem Moment ruft Charles ihr zu:\\
\frqq{}Ada! Ich habe alle Zahlen gefangen! Ich will neue!\flqq{} \\
Ada steht auf, nimmt ihre Kreide und geht zur Zahlenlinie.
\frqq{}Schau, Charles, ich zeichne dir deine eigenen Zahlen.\flqq{}\\
Ada zeichnet $1/2$ zwischen die Zahl $0$ und die Zahl $1$. Ada fängt dann an zu zählen. Um die Zahl $1$ zu erreichen, muss Charles zwei Schritte machen. Zwei halbe Schritte von Ada: $2/2 = 1$. Diese muss nicht gezeichnet werden, also macht Ada weiter. Um zwischen $1$ und $2$ anzukommen, muss Charles drei Schritte machen. Ada schreibt $3/2$. Um zur 2 zu kommen, muss Charles vier Schritte machen: $4/2 = 2$. Ein weiterer Schritt, den Ada nicht zeichnen muss. Zwischen $2$ und $3$ schreibt Ada $5/2$ usw. Zwischen $3$ und $4$ steht $7/2$. Zwischen $4$ und $5$ steht $9/2$. Ada schreibt weiter bis $10$. 
Charles ist begeistert.\\
\frqq{}Zahlen nur für mich, danke Ada! Und wie heißen sie?\flqq{} \\
Ada denkt nach. Charles macht halbe Schritte.\\
\frqq{}Die erste Zahl, $1/2$ , heißt eine halbe. Danach folgt $3/2$, das sind drei Halbe. $5/2$, das sind fünf Halbe usw.\flqq{} \\
Charles bemüht sich nun, einen Schritt auf jede seiner eigenen Zahlen zu setzen.
Ada fragt sich, wie man $1$ in drei ($1/3$) oder $1$ in vier ($1/4$) geteilt nennt. Was ist mit den anderen: $1/5$, $1/6$, $1/7$, $1/8$ und $1/9$? Für Ada wird $1/10$ nun ein Spatz genannt. Und $1/100$ eine Ameise. 
Diese neuen Zahlen sind seltsam, denkt Ada, denn wenn man $1$ durch eine größere Zahl schneidet, wird das Ergebnis kleiner.
$1/100$ ist kleiner als $1/10$. $1/1000$ ist kleiner als $1/100$. $1/100000$ ist noch kleiner! Ada stellt sich vor, dass sie auf die Jagd nach der kleinsten Zahl geht und dabei winzig klein wird, kleiner als ein Spatz, kleiner als eine Ameise, unendlich klein.
