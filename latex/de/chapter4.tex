Ada ist stolz, denn sie hat herausgefunden, wie man Zahlen größer als $9$ schreibt, und sie hat unendlich viele Zahlen entdeckt. 
Aber Ada ist auch ein bisschen traurig, denn auf ihrem Hof kann sie nur mit ein paar Zahlen spielen. Und die sind auch noch klein: Die meisten haben nur zwei Ziffern!
Als Emilie das sieht, fragt sie sie, ob sie noch mit Addieren und Subtrahieren spielen will. 
Aber das macht Ada nicht mehr so viel Spaß. Sie hat das Gefühl, dass es nichts mehr zu entdecken gibt. 
Emilie schlägt vor, einen Snack zu holen. Aber bevor sie geht, schreibt sie noch eine neue Formel ins Heft: $3 - 5 = ~$ und zeigt sie Ada. 
So eine kleine Formel, denkt Ada, das kann doch nicht so kompliziert sein. 
Ich werde sie lösen (sagt Emilie), bevor Emilie zurückkommt. 
Ada stellt sich auf den $3$ und beginnt rückwärts zu gehen: ein Schritt, zwei Schritte, drei Schritte und...\\
\frqq{}Oh nein, schon wieder!\flqq{}\\
Ada steht nun auf dem $0$, vor der Haustür, am Anfang der Linie. 
Sie muss noch zwei Schritte gehen, um ihre Formel zu beenden, aber die Linie bleibt stehen. 
Ada nimmt ihre Kreide und setzt die Linie hinter dem Haus bis zur Garage fort. 
Sie sieht, dass sie zwei Schritte zurückgehen muss, um ihre Formel zu beenden. 
Es ist genug Platz, um mindestens zwanzig Schritte weiter zu gehen, denkt Ada. 
Das Problem ist, dass auf dieser Seite der Linie keine Zahl steht. 

Ada denkt nach. 
Sie überlegt, ob sie die Zahlen, die sie schon kennt, wieder verwenden kann. Ausgehend von vor dem Haus ($0$) geht sie einen Schritt zurück und notiert $1$, dann geht sie zwei Schritte zurück und notiert $2$. 
Das Problem ist, denkt Ada, dass man den Unterschied zwischen $1$ hinter dem Haus und $1$ vor dem Haus im Hof nicht erkennen kann. 
Sie überlegt, dass sie Farbe benutzen könnte. 
Grün für vorne und orange für hinten. 
Farbe ist gut, aber Ada hat noch ein anderes Problem. 
Wenn sie rückwärts geht, d.h. wenn sie eins wegnimmt, werden die Zahlen immer größer. Und wenn sie vorwärts geht, d.h. wenn sie eins addiert, werden die Zahlen immer kleiner.
Das ist genau umgekehrt! Wir können nicht auf der einen Seite des Hauses Regeln haben und auf der anderen Seite andere. 
Am Ende bringen wir alles durcheinander, denkt Ada. 
Sie überlegt weiter, schaut in ihr Heft mit all den Formeln und überlegt: Wenn $-1$ draufsteht, heißt das, dass ich einen Schritt zurück mache. Wenn $-2$ steht, dann nehme ich zwei weg und gehe zwei Schritte zurück. Wenn $-3$ steht, bedeutet das, dass ich drei wegnehme und drei Schritte zurück gehe. 
Ada steht wieder vor dem Haus bei $0$. Sie geht einen Schritt zurück und schreibt auf den Boden $-1$ (Ich bin von Null einen Schritt zurückgegangen). Sie geht wieder einen Schritt zurück und schreibt $-2$ (Ich bin seit Null zwei Schritte zurückgegangen). Dann zählt sie die Schritte weiter und schreibt auf den Boden: $-3$, $-4$, $-5$ usw. 
Bis sie schließlich bei der Garage ankommt, wo Ada $-21$ aufschreibt. 
In diesem Moment kommt Emilie mit dem Snack zurück.\\
\frqq{}Na, Ada, du hast die negativen Zahlen entdeckt\flqq{}, staunt ihre Tante.\\
Ada ist stolz auf sich. Emilie erklärt ihr, dass auch die negativen Zahlen weit gereist sind und dass sie lange Zeit als seltsame Zahlen galten. Ada denkt sich, dass sie gar nicht seltsam sind, und dann kann sie jetzt dank ihnen Emilies Formel beenden. Ada stellt sich auf den $3$ und geht fünf Schritte zurück. Jetzt ist sie auf $-2$ angekommen.\\
\frqq{}So\flqq{}, sagt Ada, \frqq{}$3 - 5 = -2$. Dürfen wir jetzt essen?\flqq{}

%\begin{figure}[!h]
%    \centering
%    \begin{tikzpicture}[thick,scale=0.6, every node/.style={scale=0.6}, every arrow/.style={scale=0.6}]
%    
\draw (-8,-3) -- (8,-3) -- (8,10) -- (-8,10) -- (-8,-3);
\draw (-1,0) -- (1,0) -- (1,2) -- (0,3) -- (-1,2) -- (-1,0)
(-1,2) -- (1,2);
\draw (0,-1) -- (7.5,-1);
\draw[dashed] (0,-1) -- (-2,-1);
\node at (0,-0.5) {0};
\node at (1,-0.5) {1};
\node at (2,-0.5) {2};
\node at (3,-0.5) {3};
\node at (4,-0.5) {4};
\node at (5,-0.5) {5};
\node at (6,-0.5) {6};
\node at (7,-0.5) {7};
%    \end{tikzpicture}
%    \caption{Comme Ada, continue la ligne de l'autre côté de la maison. Complete le dessin comme tu l'entends}
%    \label{fig:}
%\end{figure}