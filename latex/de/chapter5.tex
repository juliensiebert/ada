Charles, Adas kleiner Bruder, ist zu seiner Schwester und Emilie zum Snack gekommen. Nachdem sie ein paar Kekse gegessen haben, fragt Ada Emilie: \\
\frqq{}Sag mal, Tante, kannst du eigentlich hüpfen? Schau mal!\flqq{}\\
Ada stellt sich auf der $0$, macht einen ersten Sprung und kommt auf der $2$. Ein zweiter und sie landet auf der $4$. Ein weiterer und Ada landet auf der $6$.\\
\frqq{}Schau, Emilie, bei jedem Sprung komme ich zwei Schritte weiter! Jetzt bist du dran!\flqq{}\\
Emilie stellt sich ebenfalls auf der $0$, macht einen ersten Sprung und kommt auf der $3$. Ein weiterer Sprung und sie ist ebenfalls auf der $6$.\\
\frqq{}Jetzt bin ich dran!\flqq{}, rief Charles.\\
Und auch er beginnt zu springen. Von der $0$ aus landet er auf der Zahl $1$. Dann auf der $2$ und $3$ usw., bis er nach sechs Sprüngen mit Emilie und Ada auf der Zahl $6$ landet.
Ada fragt sich nun, wo sie landen würde, wenn sie zehn Sprünge hintereinander machen würde. 
Sie beginnt wieder von vorne ($0$). Nach drei Sprüngen landet sie wieder auf der $6$. 
Nach einem vierten Sprung landet sie auf der $8$. 
Nach einem fünften Sprung ist sie wieder auf der $10$. Sie macht so weiter und zählt die Anzahl der Sprünge und kommt nach zehn Sprüngen auf $20$ an. \\
\frqq{}Jetzt bist du dran\flqq{}, sagt Ada zu Emilie.\\
Emilie steht auf der Zahl $0$ und macht einen Sprung (sie landet auf der $3$), dann einen zweiten (sie ist auf der $6$) und einen dritten (auf der $9$) usw. Nach dem zehnten Sprung ist sie viel weiter weg als Ada, nämlich bei der Zahl $30$.
Ada ist erstaunt, denn sie und Emilie haben jeweils zehn Sprünge gemacht. Da Emilies Sprünge aber größer sind als Adas, ist Emilie viel weiter gekommen als Ada.\\ 
\frqq{}Und nun bist du dran, Charles!\flqq{}, sagte Ada zu ihrem Bruder.\\
Und Charles beginnt, zehnmal hintereinander zu springen. Ausgehend von der $0$ kommt er schließlich zur $10$.
Ada möchte sich alle diese Zahlen merken. Sie nimmt ihr Notizbuch und schreibt auf:
\begin{description}[leftmargin=0.5cm]
    \item $2+2+2+2+2+2+2+2+2+2=20$ für sie,
    \item $3+3+3+3+3+3+3+3+3+3=30$ für Emilie und
    \item $1+1+1+1+1+1+1+1+1+1=10$ für Charles.
\end{description}
Wie lang und kompliziert, denkt Ada. 
Wenn wir 50 oder 100 Sprünge machen müssten, hätten wir nicht genug Platz, um alles aufzuschreiben! 
Wir bräuchten einen schnelleren Weg, um aufzuschreiben, dass wir zehnmal das Gleiche gemacht haben. Dann bräuchten wir nur zwei Zahlen: z.B. die Anzahl der Sprünge ($10$) und die Größe jedes Sprungs ($2$). Das wäre noch praktischer. 
Ada zeigt Emilie ihr Notizbuch und fragt sie, ob es nicht einen einfacheren Weg gäbe, $2+2+2+2+2+2+2+2+2=20$ zu schreiben. Emilie zeigt ihm ein neues Symbol, einen kleinen Punkt, das \qq{mal} heißt und so geschrieben wird: $\cdot$. Emilie zeigt ihr, dass $2+2+2+2+2+2+2+2=20$ als $10 \cdot 2 = 20$ geschrieben werden kann und dass man zehnmal zwei sagt.
Ada schaut den Neuankömmling an und schreibt $10 \cdot 3 = 30 $ in ihr Heft. 
\\
\frqq{}Das ist immer noch praktischer, als $3+3+3+3+3+3+3+3=30$ zu schreiben\flqq{}, sagt sie, \frqq{}und das ist genau das, was ich gebraucht habe. Und für dich, Charles, schreiben wir $10 \cdot 1 = $10.\flqq{}\\
Etwas beunruhigt Ada jedoch. 
\frqq{}Emilie\flqq{}, fragt sie, \frqq{}du hast mir gesagt, wenn man $+$ schreibt, macht man eine Addition. Wenn wir $-$ schreiben, machen wir eine Subtraktion. Was machen wir, wenn wir $\cdot$ schreiben?\\
\frqq{}Das nennt man Multiplikation\flqq{}, antwortet Emilie, \frqq{}man sagt, dass man zwei Zahlen miteinander multipliziert.\\
\frqq{}Wenn ich also $50$ Sprünge mache, auf welche Zahl komme ich dann?\flqq{}\\
\frqq{}Auf der Zahl 100, denn $50 \cdot 2 = 100$.\flqq{}\\
\frqq{}Was ist, wenn ich $100$ mache?\flqq{}\\
\frqq{}Auf der Zahl 200, denn $100 \cdot 2 = 200$.\flqq{}\\
Ada stellt sich vor, dass sie sehr müde wäre, wenn sie so viel springen würde.
