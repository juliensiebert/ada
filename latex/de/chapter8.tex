Emilie ist wieder da. Leider ohne Süßigkeiten. Das ist schade, denkt Ada, aber egal, Ada muss Emilie erst einmal etwas zeigen. Das mit den Süßigkeiten werden wir später sehen.\\
\frqq{}Sieh mal, Emilie\flqq{}, sagt Ada, \frqq{}Charles macht halbe Schritte. Wenn ich einen Schritt mache, muss er zwei machen.\flqq{}\\
\frqq{}Ach ja? Und wie viel ist ein Schritt von Emilie in Schritten von Ada wert, fragt ihre Tante?\flqq{}\\
\frqq{}Ich weiß es nicht, aber wir müssen es einfach versuchen\flqq{}, antwortet Ada. \frqq{}Ich habe eine Idee. Wir stellen uns nebeneinander, gehen zusammen und schauen, wie viele Schritte wir machen müssen, um an denselben Ort zu kommen.\flqq{}\\
Ada und Emilie setzen sich beide vor das Haus auf $0$. Ada macht einen Schritt und kommt auf $1$. Emilie macht ebenfalls einen Schritt, aber ihre Schritte sind größer und sie ist zwischen $1$ und $2$ eingeklemmt.
Ada macht einen zweiten Schritt und geht an Emilie vorbei. Emilie macht einen zweiten Schritt und kommt auf $3$.  Ada muss einen dritten Schritt machen, um zu Emilie zu gelangen.
Ada notiert nun in ihrem Notizbuch $3$ Ada $= 2$ Emilie. Drei Schritte von Ada sind zwei Schritte von Emilie. Das hört sich kompliziert an, denkt Ada. Ich möchte lieber wissen, wie viele Schritte ein Schritt von mir in Emilys Schritten macht. Wenn drei meiner Schritte zwei Schritte von Emilie ergeben, dann erhalte ich durch Dreiteilung einen Schritt von Ada, der zwei Schritte von Emilie entspricht, wenn ich ihn dreiteile. Ada notiert $1$ Ada $= 2/3$ Emilie.
Emilie schaut auf das Notizbuch und ruft aus: \\
\frqq{}Oh, die schönen Gleichungen!\flqq{}\\
\frqq{}Die Gleich-was?\flqq{}, wundert sich Ada.\\
\frqq{}Die Gleichungen\flqq{}, fährt Emilie fort. \frqq{}Das sind die Formeln, die du gerade aufgeschrieben hast. So werden sie genannt. Sie sind wie Rezepte, die uns sagen, wie wir Dinge umwandeln können.\flqq{}\\
\frqq{}Ich wusste es\flqq{}, sagt Ada, \frqq{}man kann Zahlen und Buchstaben mischen.\flqq{}\\
\frqq{}Ja, man kann\flqq{}\\
\frqq{}Ich frage mich, wie viele Schritte du machen musst, wenn ich bis zur 12 gehe\flqq{}, sagt Ada.\\
\frqq{}Naja, wir könnten es versuchen\flqq{}, antwortet Emilie. \frqq{}Ich stelle mich auf die $0$. Ich gehe ein, zwei Schritte weiter und stehe auf $3$. Nach drei und vier Schritten stehe ich auf $6$. Fünf und sechs Schritte sind auf $9$. Sieben und acht Schritte. Und das war's! Ich bin auf $12$. Acht Schritte, das ist deine Antwort, Ada.\flqq{}\\
\frqq{}He, lasst mich nicht allein!", rief Charles plötzlich aus dem Haus.\flqq{}\\
\frqq{}Du bist nicht allein\flqq{}, antwortet Ada, \frqq{} außerdem bist du nicht einmal weit weg.\flqq{}\\
\frqq{}Doch, ich bin sehr weit weg, mindestens hundertachtundfünfzig!\flqq{}\\
\frqq{}Unsinn\flqq{}, sagt Ada, \frqq{}wir sind bei $12$ und du bist auf $0$. Da du halbe Schritte machst, musst du die doppelte Anzahl an Schritten machen wie ich. Ada rechnet im Kopf, vierundzwanzig, stimmt's, Emilie?\flqq{}\\
\frqq{}Ja, das ist richtig. Komm, Charles, es ist nicht so weit weg.\flqq{}\\
Charles rennt zu ihnen.\\ 
\frqq{}Und jetzt\flqq{}, sagt Emilie, \frqq{}muss ich in den Garten gehen, um Salat zu holen. Wer kommt mit mir?\flqq{}\\
\frqq{}Ich!\flqq{}, ruft Charles.\\
\frqq{}Und du, kommst du auch mit, Ada?\flqq{}\\
\frqq{}Ich weiß nicht, der Salat ist ganz hinten im Garten und das kommt mir weit weg vor. Es wäre schön\flqq{}, fährt Ada fort, \frqq{}wenn wir eine Methode hätten, mit der wir wissen, wie viele Schritte ich machen muss, um dich zu erreichen. Am besten wäre es, wenn ich die Antwort sofort bekäme, wenn ich einfach nur schaue, ohne zu rechnen.\flqq{}\\
\frqq{}Ich kann dir die Gleichungen zeichnen\flqq{}, schlägt Emilie vor.\\
\frqq{}Aber ich habe sie schon in mein Notizbuch geschrieben!\flqq{}\\
\frqq{}Nein, schau.\flqq{}\\
Emilie zeichnet zwei Linien in das Heft: eine horizontale für ihre eigenen Schritte und eine vertikale für Adas Schritte. Emilie notiert dann $0$ an der Stelle, an der sich die Linien kreuzen, und fügt dann auf jeder Linie Zahlen von $1$ bis $9$ hinzu. Sie zeigt Ada, wie sie die Gleichung zeichnen soll. Wenn Emilie zwei Schritte macht, muss Ada drei Schritte machen. Emilie legt den Stift auf die Nummer $2$ der waagerechten Linie und steigt bis zur Nummer $3$ der senkrechten Linie. Dort markiert sie einen ersten Punkt. Wenn Emilie vier Schritte macht, muss Ada sechs Schritte machen. Auf die gleiche Weise macht Emilie einen zweiten Punkt. Wenn Emilie sechs Schritte macht, muss Ada neun Schritte machen. Schließlich markiert Emilie einen dritten Punkt auf dem Blatt. Dann zieht sie eine Linie, die alle Punkte verbindet.\\
\frqq{}Siehst du\flqq{}, sagt Emilie, \frqq{}diese Linie, die ich gerade gezogen habe, ist meine Gleichung. Wenn du wissen willst, wie viele Schritte du gehen musst, um mich zu erreichen, musst du nur dieser Linie folgen.\flqq{}\\
\frqq{}Der Linie folgen?\flqq{}\\
\frqq{}Ja, schau dir das an. Wenn ich acht Schritte mache, sagt mir die Linie, dass du zwölf Schritte machen musst.\flqq{}\\
\frqq{}Das wusste ich schon!\flqq{}, antwortet Ada.\\
\frqq{}Wenn ich zehn Schritte mache, musst du fünfzehn machen.\flqq{}\\ 
\frqq{}Und wenn du nur fünf machst?\flqq{}, fragt Ada.\\
\frqq{}Sie sagt, du musst siebeneinhalb machen.\flqq{}\\
Ada schaut sich die Zeichnung in ihrem Heft an. Das gefällt ihr gut. Nach den Zaubersprüchen und Rezepten ist das hier eine Schatzkarte.
