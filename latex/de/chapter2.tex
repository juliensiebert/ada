Ada spielt weiter mit Additionen und Subtraktionen. 
Emilie hat ihr ihr Heft und ihren Stift dagelassen und ist dann in den Garten gegangen, um Blumen zu pflücken. 
Nach einigen Spielrunden hat Ada ein Problem. 
Sie hat nicht genug Zahlen! 
Sie hat $4 + 5 + 2 - 3$ im Notizbuch notiert, sich auf die $4$ gestellt, fünf Schritte vorwärts gemacht und schon sitzt sie in der Klemme. 
Sie steht nun auf $9$, der letzten Zahl in der Reihe. 
Ada muss noch zwei Schritte nach vorne gehen und dann drei Schritte in die andere Richtung machen.
Aber siehe da: Nach $9$ hat Ada nichts mehr gezeichnet!
Sie könnte auf der Linie vor- und zurückgehen, ohne Zahlen zu haben, wo sie ihre Füße hinsetzen soll. Aber woher weiß sie, ob sie an der richtigen Stelle ankommt? 
Sie weiß, dass es Dinge nach $9$ gibt. 
Sie hat schon von Zahlen gehört, z. B. einundzwanzig, dreiundsechzig, sechsunddreißig oder fünfzehn. 
Nur weiß Ada nicht, wie man sie zeichnet, und auch nicht, in welcher Reihenfolge.
Wer kommt zuerst? einundzwanzig oder fünfzehn? dreiundsechzig oder sechsunddreißig?
Sie ist sehr besorgt. Was soll sie tun? Soll sie ihre Tante Emilie fragen? Nein, Emilie hat bestimmt viel zu tun, und Ada spürt, dass sie es auch allein schaffen kann. Also denkt sie nach. Sie überlegt, ob sie sich nicht neue Zahlen ausdenken könnte. 
Sie müsste nur neue Symbole zeichnen und ihnen neue Namen geben. Leichter gesagt als getan. Nach $4$ neuen Zahlen, \qq{ga}, \qq{bu}, \qq{zo} und \qq{meu}\footnote{Anm. des Autors: Diese Zahlen wurden nicht von Ada erfunden, sondern stammen aus einer französischen Zeichentrickserie namens Les Shadoks, siehe \url{https://de.wikipedia.org/wiki/Die_Shadoks}.}, gehen Ada die Ideen aus. 
Das ist gar nicht so einfach. Man muss neue Namen finden und sich neue Symbole zum Zeichnen ausdenken (wer hat übrigens die Zahlen von $0$ bis $9$ erfunden? fragt sich Ada). Außerdem ist es schwierig, sich solche Dinge zu merken. Schon das Merken der Namen der Zahlen von $0$ bis $9$ hatte ihr Schwierigkeiten bereitet, stellt euch vor, wenn sie sich für jede neue Zahl einen neuen Namen und ein neues Symbol merken müsste.
Es könnte ja jede Menge Zahlen geben! 
Also, nein, Ada beschließt, dass es vielleicht doch keine so gute Idee ist, neue Zahlen zu erfinden. 
Was ist zu tun? Ada, die immer noch auf dem $9$ steht, muss noch zwei Schritte nach vorne und drei rückwärts gehen. Sie überlegt noch einmal. Vielleicht kann sie die Zahlen wieder benutzen? 
Nach $9$ kann man wieder $0$, $1$, $2$ usw. schreiben. So muss man keine neuen Symbole lernen. Das ist gut, denkt Ada, aber das Problem ist, dass man sich verlaufen kann. 
Woher weiß ich, wie weit ich von zu Hause entfernt bin? Wenn ich auf einem $5$ bin, bin ich dann auf dem ersten $5$ - dem direkt neben dem Haus - bin ich dann auf dem zweiten? dem dritten? Wie soll ich mich orientieren, fragt sich Ada.
Sie denkt weiter. Was wäre, wenn sie Farbe benutzen würde? 
Eine Farbe für die erste Zahlenreihe von $0$ bis $9$, z. B. grün. Dann eine andere Farbe, sagen wir Gelb, für die zweite Zahlenreihe, dann Rot, dann Blau. Hmm, ist Blau größer als Rot? Oder wäre es besser, Blau für die dritte Zahlenreihe und Rot für die vierte zu verwenden? Und welche Farbe für die fünfte? Ada denkt, dass diese Lösung mehr Probleme aufwirft als sie löst.  
Die Farben sind vielleicht nicht die richtige Lösung, aber all das bringt sie auf eine andere Idee. Ada notiert, wie oft sie alle Zahlen verwendet. Sie beginnt von zu Hause aus: $0$, $1$, $2$, $3$, $4$, $5$, $6$, $7$, $8$ und $9$. Nach $9$ beginnt Ada die Zahlenreihe erneut. Sie notiert $0$. Und damit sie nicht vergisst, dass sie gerade die ganze Zahlenreihe zum ersten Mal benutzt hat, notiert sie links von sich einen $1$. Ada erhält $1~0$ (\qq{eins-null}). Und sie fährt fort, nachdem sie $1~1$ (\qq{eins-eins}), $1~2$ (\qq{eins-zwei}), $1~3$ (\qq{eins-drei}), $1~4$, $1~5$, $1~6$, $1~7$, $1~8$ und $1~9$ notiert hat. 
Dort angekommen, verlängert Ada die Linie der Zahlen auf dem Weg zum Garten. 
Da sie die gesamte Zahlenreihe ein zweites Mal verwendet hat, notiert sie nun $2~0$ (\qq{zwei-null}) und fährt fort: $2~1$ (\qq{zwei-eins}), $2~2$ (\qq{zwei-zwei}), $2~3$, $2~4$, $2~5$. 
Ada hält einen Moment inne und schaut sich ihre Lösung an. Es gefällt ihr. Erstens ist man nie verloren, denkt sie (man weiß immer, ob man weit weg oder nah am Haus ist), und zweitens kann man die Zahlen in jeder beliebigen Farbe zeichnen! 
Emilie, die in diesem Moment aus dem Garten zurückkommt, sagt zu ihr:\\
\frqq{}Na, was machst du denn da auf der Fünfundzwanzig?\flqq{}\\
\frqq{}Die fünfundzwanzig ?\flqq{}, fragt Ada, \frqq{}heißt die \qq{zwei-fünf} eigentlich fünfundzwanzig?\flqq{}\\
\frqq{}Ja\flqq{}, antwortet Emilie, \frqq{}alle Zahlen haben Namen.\flqq{}\\
%\frqq{}Les nombres ?\flqq{} demande Ada.\\
%\frqq{}Oui, dit Émilie, c’est comme ça qu'on les appelle. Les chiffres sont les symboles : $0$, $1$, $2$, $3$, $4$, $5$, $6$, $7$, $8$ et $9$. Il n’y en a que dix. Avec les chiffres on peut écrire des nombres : comme, par exemple, vingt-cinq avec un deux et un cinq : $25$.\flqq{}\\
\frqq{}OK, und wie heißt die mit einer Eins und einer Vier?\flqq{} fragt Ada.\\
\frqq{}Sie heißt vierzehn.\flqq{}\\
\frqq{}Und die da? Die seltsame mit einer 1 und einer 7?\flqq{}\\
\frqq{}Diese Zahl heißt siebzehn.\flqq{}\\ 
Emilie bringt Ada die Namen der Zahlen bei: \qq{zehn} ($10$), \qq{elf} ($11$), \qq{zwölf} ($12$), usw. bis $25$, wo Ada aufgehört hatte. 
Ada fragt sich, wer über die Namen der Zahlen entscheidet. Emilie erklärt ihr, dass die Zahlen weit gereist sind: China, Indien, Mittlerer Osten, Zentralasien, Nordafrika, Europa. 
Die Zahlen, wie wir sie kennen, werden überall auf der Welt verwendet.
Ada findet das sehr praktisch. Sie stellt sich eine Welt vor, in der die Zahlen ihre Form ändern, wenn man von einem Land in ein anderes reist. Wie kompliziert das wäre!

%\newpage
%\vspace*{\fill}
%\begin{figure}
%    \centering
%    \begin{tikzpicture}[thick,scale=0.6, every node/.style={scale=0.6}, every arrow/.style={scale=0.6}]
%    % Grid
\draw[lightgray!20] (0,0) grid (12,12);

% Puzzle
\draw[line width=3pt, draw=black!75] (5,12) -- (0,12) -- (0,0) -- (6,0) -- (6,4) -- (7,4) -- (7,5) -- (9,5) -- (9,7) -- (10,7)
(7,0) -- (12,0) -- (12,12) -- (6,12)  -- (6,9) -- (3,9)
(1,0) -- (1,4)
(1,6) -- (1,5) -- (2,5) -- (2,2)
(2,3) -- (3,3) -- (3,1)
(2,1) -- (5,1) -- (5,2) -- (4,2)

(6,3) -- (5,3)
(2,5) -- (3,5) -- (3,4) -- (4,4) -- (4,3)
(4,4) -- (5,4) -- (5,5) -- (6,5) -- (6,6) -- (7,6) -- (7,7) -- (8,7) -- (8,8) -- (12,8)
(8,6) -- (9,6)
(6,1) -- (9,1) -- (9,4) -- (8,4) -- (8,3) -- (7,3) -- (7,2) -- (8,2)
(10,0) -- (10,1)
(11,1) -- (12,1)
(9,2) -- (11,2)
(10,3) -- (12,3)
(11,3) -- (11,7)
(9,4) -- (10,4) -- (10, 6)
(0,7) -- (2,7) -- (2,6) -- (4,6) -- (4,5)
(5,5) -- (5,7) -- (3,7)
(4,7) -- (4,8)
(0,8) -- (1,8)
(2,7) -- (2,9) -- (1,9)
(3,8) -- (3,10) -- (1,10) -- (1,11)
(2,12) -- (2,11)
(3, 11) -- (3,10) -- (5,10) -- (5,11)
(4,12) -- (4,11)
(5,9) -- (5,8)
(6,7) -- (6,8) -- (8,8) -- (8,9)
(6,9) -- (7,9)
(7,11) -- (7,10) -- (9,10) -- (9,9) -- (10,9) -- (10,8)
(8,11) -- (10,11) -- (10,10) -- (11,10) -- (11,9)
(10,12) -- (10,11)
(12,11) -- (11,11);

% Start and End Points
\draw[-latex,line width=3pt,red] (6.5,-1) -- (6.5,0);
\draw[-latex,line width=3pt,red] (5.5,12) -- (5.5,13);


% numbers all around (TODO Foreach loop)

\edef\n{1}
\foreach \x in {5.5,5.0,...,-0.5} {
    \node at (\x,-0.5) {\n};
    \pgfmathparse{int(Mod(\n+1,10))}
    \xdef\n{\pgfmathresult}
}

\foreach \y in {0.0,0.5,...,12.5} {
    \node at (-0.5,\y) {\n};
    \pgfmathparse{int(Mod(\n+1,10))}
    \xdef\n{\pgfmathresult}
}

\foreach \x in {0.0,0.5,...,4.5} {
    \node at (\x,12.5) {\n};
    \pgfmathparse{int(Mod(\n+1,10))}
    \xdef\n{\pgfmathresult}
}

\edef\n{1}
\foreach \x in {7.5,8.0,...,12.5} {
    \node at (\x,-0.5) {\n};
    \pgfmathparse{int(Mod(\n+1,10))}
    \xdef\n{\pgfmathresult}
}

\foreach \y in {0.0,0.5,...,12.5} {
    \node at (12.5,\y) {\n};
    \pgfmathparse{int(Mod(\n+1,10))}
    \xdef\n{\pgfmathresult}
}

\foreach \x in {12.0,11.5,...,6.5} {
    \node at (\x,12.5) {\n};
    \pgfmathparse{int(Mod(\n+1,10))}
    \xdef\n{\pgfmathresult}
}
%    \end{tikzpicture}
%    \caption{Ada est perdue au milieux des chiffres. Aide la à retrouver sa maison.}
%\end{figure}
%\vspace*{\fill}