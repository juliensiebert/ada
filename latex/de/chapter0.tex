Das ist Ada. Heute hat sie beschlossen, draußen mit ihren Kreiden zu spielen. 
Sie möchte die Zahlen malen, die sie gerade in der Schule gelernt hat. Sie sind lustig mit ihren seltsamen Formen und Namen.
Als Erstes zieht Ada eine Linie. Sie geht vom Haus über den Hof in den Garten.
Dort sollen die Zahlen stehen. Sie dürfen nicht herunterfallen! 
Ada malt zunächst einen Kreis in der Nähe ihres Hauses. Das ist die $$0$$. Das ist der Startpunkt. 
Dann macht sie einen Schritt in Richtung Garten und folgt der Linie. Ada malt nun eine $1$. Das ist einfach, es sieht aus wie ein senkrechter Balken. 
Ein zweiter Schritt in Richtung Garten und sie malt eine $2$. Diese ist etwas komplizierter: Sie dreht sich, geht nach oben und nach unten. Und natürlich darf der kleine waagerechte Balken darunter nicht fehlen.
Noch ein Schritt und Ada malt eine $3$. Zwei Kurven und fertig.
Es folgen $4$ (ganz gerade), $5$ (eine Mischung aus Geraden und Rundungen), $6$ (ganz rund), $7$ (wie ein unfertiger Zickzack), $8$ (wie zwei Kreise) und die letzte: $9$ (Achtung, nicht mit einer $6$ verwechseln). 

Ada steht jetzt in der Mitte des Hofes. Sie geht einen Schritt zurück und kommt von der $9$ auf die $8$. Noch einen Schritt zurück und schon ist sie auf der $7$, dann auf der $6$, der $5$, der $4$, der $3$, der $2$, der $1$ und der $0$, direkt vor der Haustür.\\ 
\frqq{}Das ist lustig\flqq{}, denkt sie, \frqq{}wenn ich einen Schritt in Richtung Garten mache, trete ich auf die größere Zahl. Und wenn ich einen Schritt zurück mache, trete ich auf die kleinere Zahl. 
Mal sehen, was passiert, wenn ich drei Schritte vorwärts gehe.\flqq{}\\
Ada stellt sich auf die $0$ und zählt im Gehen: \frqq{}$1$, $2$ und $3$!\flqq{}\\
Wieder drei Schritte: \frqq{}$4$, $5$ und $6$!\flqq{}\\
Und noch einmal: \frqq{}$7$, $8$ und $9$!\flqq{}\\
\frqq{}Und jetzt\flqq{}, sagt Ada, \frqq{}in die andere Richtung!\flqq{} Mal sehen, was passiert, wenn ich zwei Schritte zurücktrete : \\
\frqq{}$8$ und $7$!\flqq{}\\ 
\frqq{}$6$ und $5$!\flqq{}\\ 
\frqq{}$4$ und $3$!\flqq{}\\ 
\frqq{}$2$ und $1$!\flqq{}\\
\frqq{}Ich bin fast da\flqq{}, denkt Ada.
