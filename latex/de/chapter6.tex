Ada hat noch ein bisschen Hunger.
Es gibt keine Keckse mehr, aber zum Glück hat Emilie Himbeeren aus dem Garten mitgebracht.
Es sind zwölf Stück. Ada überlegt, wie sie sie aufteilen soll. 
Denn auch Emilie und Charles mögen Himbeeren. 
Ada überlegt. Ich gebe zuerst eine für Charles, eine für Emilie und eine für mich. 
Es bleiben (Ada zählt laut die Himbeeren) neun übrig. 
Hier ist eine für Charles, eine für Emilie und eine für mich. Jetzt sind nur noch sechs übrig. 
Hier ist eine dritte für Charles, eine für Emilie und eine für mich. Jetzt sind nur noch drei übrig. 
Zum Schluss gibt Ada jedem noch eine vierte Himbeere. Jetzt sind alle zwölf Himbeeren aufgeteilt.\\
\frqq{}Wer hat noch Hunger? Wer will Himbeeren?\flqq{}, fragt Ada. \frqq{}Schau, wir haben alle vier.\frqq{}\\
\frqq{}Bravo", sagte Emilie, "du kennst auch die Division!\flqq{}\\
\frqq{}Die was?\flqq{}, wundert sich Ada.\\
\frqq{}Die Division\flqq{}, erklärt Emilie. \frqq{}Das hast du gerade gemacht, um die Himbeeren zu teilen. Die Division ist die Freundin der Multiplikation, der Subtraktion und der Multiplikation.\flqq{}\\
\frqq{}Ach ja?\flqq{}, antwortet Ada mit dem Mund voller Früchte. \frqq{}Und wie zeichnet man die?\flqq{}\\
\frqq{}Oft wird ein Strich verwendet, um zu zeigen, dass man eine Zahl durch eine andere teilt (schneidet).\flqq{}\\ 
\frqq{}Wir schneiden die Zahlen ab!\flqq{}, schreit Ada.\\
Ada stellt sich Zahlen vor, die in zwei Hälften geteilt sind. Ein $8$, der seine obere Hälfte verloren hat (\clipbox{0 -0.5 0 {0.4\totalheight}}{$8$}) oder eine $4$ ohne Fuß (\clipbox{0 {0.3\totalheight} 0 0}{$4$}). Komisch!\flqq{}\\
\frqq{}Und das tut ihnen nicht weh?\flqq{}\\
\frqq{}Aber nein\flqq{}, beruhigt Emilie sie. \frqq{}Hier, schau mal.\flqq{}\\
Emilie nimmt das Heft und notiert: $12/3=4$.\\
\frqq{}Zwölf Himbeeren, die durch drei geteilt werden, entsprechen vier Himbeeren pro Person. Man kann es auch so schreiben.\flqq{}\\
Emilie reicht Ada das Heft und zeigt ihr: 
$$\frac{12}{3}=4$$
Ada findet die Division sehr schön. Sie fragt sich, ob es für alle Zahlen funktioniert und nicht nur für Himbeeren. 
Sie möchte wissen, wie viel $7$ geteilt durch $3$ ergibt. Wie macht man das? Bei den Himbeeren war es ganz einfach: Man musste sie nur aufteilen. Eine für Ada, eine für Charles, eine für Emilie, zwei für Ada, zwei für Charles, zwei für Emilie und so weiter, bis keine Himbeeren mehr übrig waren. 
Aber Ada hat keine Himbeeren mehr. Vielleicht hat sie keine Himbeeren mehr, aber sie hat ihre Kreide und einen Strich aus Zahlen, der sich über den ganzen Hof und Garten zieht, also denkt Ada nach. Das ist es! Sie hat eine Idee. Sie wird ihre Farben benutzen.
Ada steht auf und nimmt drei Kreiden: eine grüne, eine orange und eine violette. Dann stellt sie sich auf $0$, geht drei Schritte nach vorne und umkreist $1$ grün, $2$ orange und $3$ violett. Es ist wie bei den Himbeeren: ein Schritt für grün, ein Schritt für orange und ein Schritt für lila.
Jetzt geht sie drei Schritte weiter und umkreist $4$ grün, $5$ orange und $6$ grün. Jetzt hat jede Farbe zwei Schritte.
Ada will wieder drei Schritte gehen, aber irgendetwas stimmt nicht. Sie bleibt stehen und schaut. Es ist nur noch ein Schritt bis zum $7$. Nicht genug, um jeder Farbe einen zu geben! Das geht nicht.
\frqq{}Emilie? Deine Division, sie funktioniert nicht.\flqq{}\\
\frqq{}Wirklich?\flqq{}, wundert sich Emilie.\\
\frqq{}Ja\flqq{}, sagte Ada, \frqq{}ich wollte $7$ durch $3$ teilen. Ich habe drei Päckchen: ein grünes, ein oranges und ein lilafarbenes. Ich habe in jedem Paket zwei Schritte gemacht. Aber ich muss noch einen Schritt machen, um fertig zu werden, und den kann ich nicht durch drei teilen!\flqq{}\\
\frqq{}Das ist normal\flqq{}, erklärt Emilie, \frqq{}nicht alle Zahlen lassen sich immer vollständig teilen. Manchmal gibt es einen Rest.\flqq{}\\
\frqq{}ein Rest?\flqq{}\\
\frqq{}Ja ein Rest, so nennt man das. In deinem Fall ist $7$ geteilt durch $3$ gleich $2$ und es bleibt ein Rest von $1$. Man kann auch sagen, dass $7$ gleich $3$ Pakete aus $2$ plus $1$ sind. Das ist das Gleiche. Und wir können es so notieren: $7=3\cdot2+1$.\flqq{}\\
\frqq{}Hä?\flqq{}, ruft Ada aus. \frqq{}Wo ist meine Abteilung hin? Du schummelst schon wieder, Tante!\flqq{}\\
\frqq{}Aber natürlich nicht, die Multiplikation und die Division sind so gute Freunde, dass man das eine in das andere umwandeln kann. Man sagt auch, dass die Division die Umkehrung der Multiplikation ist.\flqq{}\\
Ada schaut sie erstaunt an. Nach einiger Zeit sagt sie:\\
\frqq{}Okay, also bitte, kannst du uns sechs Bonbons holen? Zwei für jeden? Denn $6 = 3 \cdot 2$ und $6 / 3 = 2$, richtig?\flqq{} 

