Ada hat noch ein bisschen Hunger.
Es gibt keine Kekse mehr, aber zum Glück hat Emilie Himbeeren aus dem Garten mitgebracht.
Es sind zwölf Stück. Ada überlegt, wie sie sie aufteilen soll. 
Denn auch Emilie und Charles mögen Himbeeren. 
Ada überlegt. Ich gebe zuerst eine für Charles, eine für Emilie und eine für mich. 
Es bleiben (Ada zählt laut die Himbeeren) neun übrig. 
Hier ist eine für Charles, eine für Emilie und eine für mich. Jetzt sind nur noch sechs übrig. 
Hier ist eine dritte für Charles, eine für Emilie und eine für mich. Jetzt sind nur noch drei übrig. 
Zum Schluss gibt Ada jedem noch eine vierte Himbeere. Jetzt sind alle zwölf Himbeeren aufgeteilt.\\
\frqq{}Wer hat noch Hunger? Wer will Himbeeren?\flqq{}, fragt Ada. \frqq{}Schau, wir haben alle vier.\flqq{}\\
\frqq{}Bravo\flqq{}, sagte Emilie, \frqq{}du kennst auch die Division!\flqq{}\\
\frqq{}Die was?\flqq{}, wundert sich Ada.\\
\frqq{}Die Division\flqq{}, erklärt Emilie. \frqq{}Das hast du gerade gemacht, um die Himbeeren zu teilen. Die Division ist die Freundin der Multiplikation, der Subtraktion und der Addition.\flqq{}\\
\frqq{}Ach ja?\flqq{}, antwortet Ada mit dem Mund voller Früchte. \frqq{}Und wie stellt man die dar?\flqq{}\\
\frqq{}Oft wird ein Strich verwendet, um zu zeigen, dass man eine Zahl durch eine andere teilt.\flqq{}\\ 
\frqq{}Wir zerteilen die Zahlen!\flqq{}, schreit Ada.\\
Ada stellt sich Zahlen vor, die in zwei Hälften geteilt sind. Eine $8$, die ihre obere Hälfte verloren hat (\clipbox{0 -0.5 0 {0.4\totalheight}}{$8$}) oder eine $4$ ohne Fuß (\clipbox{0 {0.3\totalheight} 0 0}{$4$}). Komisch!\flqq{}\\
\frqq{}Und das tut ihnen nicht weh?\flqq{}\\
\frqq{}Aber nein\flqq{}, beruhigt Emilie sie. \frqq{}Hier, schau mal.\flqq{}\\
Emilie nimmt das Heft und notiert: $12/3=4$.\\
\frqq{}Zwölf Himbeeren, die durch drei geteilt werden, entsprechen vier Himbeeren pro Person. Man kann es auch so schreiben.\flqq{}\\
Emilie reicht Ada das Heft und zeigt ihr: 
$$\frac{12}{3}=4$$
Ada findet die Division sehr schön. Sie fragt sich, ob es für alle Zahlen funktioniert und nicht nur für Himbeeren. 
Sie möchte wissen, wie viel $7$ geteilt durch $3$ ergibt. Wie macht man das? Bei den Himbeeren war es ganz einfach: Man musste sie nur aufteilen. Eine für Ada, eine für Charles, eine für Emilie, zwei für Ada, zwei für Charles, zwei für Emilie und so weiter, bis keine Himbeeren mehr übrig waren. 
Aber Ada hat keine Himbeeren mehr. 
Ada denkt nach. Sie hat eine Idee. Sie wird ihre bunte Kreide und ein paar Kieselsteine benutzen.
Ada steht auf und nimmt drei Stücke Kreide: ein grünes, ein gelbes und ein lila. Sie malt drei Kreise auf den Boden: einen grünen, einen gelben und einen lila. Dann sucht sie sieben Kieselsteine. Zuerst legt sie einen in den grünen, einen in den gelben und einen in den lila Kreis. Es ist wie bei den Himbeeren, nur dass Ada jetzt Steine verteilt. Ada macht weiter. Es sind noch vier Steine übrig. Sie legt einen in den grünen, einen in den gelben und einen in den lila Kreis. Jetzt sind in jedem Kreis zwei Steine. Ada schaut in ihre Hand. Sie hat nur noch einen Stein übrig. Nicht genug, um in jeden Kreis einen Stein zu legen! Das geht nicht.
\frqq{}Emilie? Deine Division, sie funktioniert nicht.\flqq{}\\
\frqq{}Wirklich?\flqq{}, wundert sich Emilie.\\
\frqq{}Ja\flqq{}, sagte Ada, \frqq{}ich wollte $7$ durch $3$ teilen. Ich habe drei Kreise gemalt: einen grünen, einen gelben und einen lila. In jeden habe ich zwei Steine gelegt. Aber es ist noch ein Stein übrig, und ich kann ihn nicht durch drei teilen!\flqq{}\\
\frqq{}Das ist normal\flqq{}, erklärt Emilie, \frqq{}nicht alle Zahlen lassen sich immer vollständig teilen. Manchmal gibt es einen Rest.\flqq{}\\
\frqq{}ein Rest?\flqq{}\\
\frqq{}Ja ein Rest, so nennt man das. In deinem Fall ist $7$ geteilt durch $3$ gleich $2$ und es bleibt ein Rest von $1$. Man kann auch sagen, dass $7$ gleich $3$ mal $2$ plus $1$ sind. Das ist dasselbe. Und wir können es so schreiben: $7=3\cdot2+1$.\flqq{}\\
\frqq{}Hä?\flqq{}, ruft Ada aus. \frqq{}Wo ist meine Division hin? Du schummelst schon wieder, Tante!\flqq{}\\
\frqq{}Aber natürlich nicht, die Multiplikation und die Division sind so gute Freunde, dass man das eine in das andere umwandeln kann. Man sagt auch, dass die Division die Umkehrung der Multiplikation ist.\flqq{}\\
Ada schaut sie erstaunt an. Nach einiger Zeit sagt sie:\\
\frqq{}Okay, also bitte, kannst du uns sechs Bonbons holen? Zwei für jeden? Denn $6 = 3 \cdot 2$ und $6 / 3 = 2$, richtig?\flqq{} 

