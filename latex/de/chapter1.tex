In diesem Moment kommt Adas Tante Emilie aus dem Haus. \\
\frqq{}Hallo Ada\flqq{}, sagt Emilie, \frqq{}wie geht es dir?\flqq{}\\
\frqq{}Emilie !\flqq{} ruft Ada.\\
Ada liebt Emilie sehr. Sie ist eine ihrer Lieblingstanten. Emilie ist so stark, sie ist viel gereist und spricht drei verschiedene Sprachen. Außerdem baut sie Raketen!\\
\frqq{}Schau mal, Emilie, ich spiele mit Zahlen\flqq{}, erklärt Ada. \frqq{}Ich habe alle Zahlen von 0 bis 9 gezeichnet. Und wenn ich einen Schritt vorwärts gehe, komme ich zur nächsten. Und wenn ich einen Schritt zurück gehe, bin ich wieder bei der vorherigen.\flqq{}\\
Emilie erklärt Ada nun, dass wenn sie einen Schritt vorwärts geht, addiert sie 1 zu der Zahl, auf der sie steht, und zeichnet $+1$. Wenn sie einen Schritt zurück geht, subtrahiert sie eins von der Zahl, auf der sie stand, und zeichnet $-1$. Ada hat jetzt zwei neue Symbole, die sie benutzen kann: das Plus ($+$) für die Addition und das Minus ($-$) für die Subtraktion. Sie findet die beiden neuen Symbole lustig. 
Ada und Emilie beschließen, zusammen zu spielen. Von der $0$ aus geht Ada drei Schritte vorwärts ($+3$) und dann einen Schritt zurück ($-1$). Jetzt steht sie auf der $2$. Emilie nimmt ein Notizbuch und einen Stift und zeigt Ada, wie sie ihren Weg zeichnen soll: $0 + 3 - 1$ . Als Ada auf der $2$ angekommen ist, erklärt Emilie ihr, dass es ein Symbol namens Gleich ($=$) gibt, mit dem man sagen kann, dass man auf der $2$ angekommen ist: $0 + 3 - 1 = 2$. Ada findet, dass sich das wie eine Zauberformel anhört. Sie will es unbedingt noch einmal versuchen. \\
\frqq{}Komm schon, Tante, jetzt schreibst du die Zahlen und die lustigen Symbole in dein magisches Notizbuch und ich bewege mich auf der Linie.\flqq{} \\
Emilie zeigt ihr ihr Notizbuch, in dem steht: $4 + 3 - 2$. Ada überlegt. \\
\frqq{}Man muss auf der 4 beginnen, drei Schritte nach vorne und zwei zurückgehen, richtig?\flqq{} \\
\frqq{} Das ist richtig!\flqq{} antwortet Emilie.\\
Dann macht sich Ada auf den Weg. Von der $4$ aus geht sie drei Schritte nach vorne und kommt auf die 7. Dann dreht sie sich um und macht zwei Schritte in die andere Richtung. Ada steht nun auf der $5$. Emilie schreibt nun $4 + 3 - 2 = 5$ auf.\\
\frqq{}Jetzt bin ich dran!\flqq{}, ruft Ada.\\
Dann gibt Emilie ihr das Notizbuch und den Stift, und Ada schreibt: $1 + 7 - 3$. Emilie schaut sich das Blatt an, steht auf und geht zur $5$.\\
\frqq{}Voilà !\flqq{} sagt Emilie und trägt $1 + 7 - 3 = 5$ in das Notizbuch ein.\\
\frqq{}Was?\flqq{}, fragt Ada. \frqq{}Du schummelst! Du musst von $1$ aus $7$ Schritte nach vorne und $3$ zurück gehen.\flqq{}\\
Und Ada zeigt es ihr. Sie stellt sich auf die $1$, geht sieben Schritte vorwärts (dann ist sie bei der $8$) und drei Schritte zurück, um genau dort anzukommen, wo Emilie steht, bei der $5$. Emilie erklärt Ada nun, dass man mit dem Notizbuch und dem Stift nicht immer vor und zurück gehen muss. Man kann das Ergebnis von Operationen (so nennt Emilie die magischen Formeln mit den Zahlen, $+$ und $-$) berechnen und das Ergebnis (was hinter dem $=$ steht) erfahren, ohne sich zu bewegen. Ada ist skeptisch. Sie bittet Emilie um ein weiteres Beispiel. Emilie stellt sich auf die $0$ und schreibt $2 + 4 - 5 + 2 - 3$ in ihr Notizbuch. Ada stellt sich auf die $2$, geht vier Schritte vor, dann fünf Schritte in die andere Richtung, dreht sich wieder um, geht zwei Schritte vor und drei Schritte zurück. Puh! Ihr wird fast schwindelig. Emilie dagegen hat sich nicht bewegt. Dann schaut Ada unter ihre Füße und sieht die $0$. Emilie lächelt und schreibt $2 + 4 - 5 + 2 - 3 = 0$.\\
\frqq{}Siehst du\flqq{}, sagte sie, \frqq{}ich wusste, dass wir beide hier auf der $0$ landen würden, schon bevor du dich auf den Weg gemacht hast.\flqq{}\\
\frqq{}Das stimmt\flqq{}, sagt Ada, \frqq{}das ist praktisch bei sehr langen Formeln, bei denen einem schwindelig wird. Aber ich finde es viel lustiger, wirklich zu reisen!\flqq{}.
