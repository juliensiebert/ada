Ada macht dort weiter, wo sie aufgehört hat. 
Sie nimmt das Notizbuch und liest $4 + 5 + 2 - 3$. 
Ada stellt sich auf $4$ und geht fünf Schritte vorwärts. 
Sie steht jetzt auf dem $9$. 
Sie geht zwei Schritte weiter. Jetzt steht sie auf $11$. 
Ada dreht sich um und geht $3$ Schritte in die andere Richtung und bleibt auf $8$ stehen.
Endlich! Ada kann nun $4 + 5 + 2 - 3 = 8$ in ihr Heft schreiben.
Sie wendet sich Emilie zu und fragt sie:\\
\frqq{}Sag mal, Tante, wer ist die größte Zahl?\flqq{}\\
Emilie schaut sie an und lächelt.
\frqq{}Bisher ist die größte Zahl, die du auf deiner Linie gezeichnet hast, fünfundzwanzig.\flqq{}\\
\frqq{}Nein\flqq{}, erwidert Ada, \frqq{}du weißt schon, die größte aller Zahlen. Nach der 25 kann ich $2~6$ schreiben (Ada spricht dann \qq{zwei-sechs} und Emilie nennt ihren Namen: \qq{sechsundzwanzig}), und $2~7$ (\qq{zwei-sechs}, \qq{siebenundzwanzig}), $28$, $29$. Danach schreibe ich $30$, um zu zeigen, dass ich alle Zahlen von $0$ bis $9$ dreimal verwendet habe. Dann schreibe ich weiter: $31$, $32$, $33$ und so weiter. Siehst du, wenn die Linie weiter in den Garten, zu den Nachbarn oder über die Straße führt, sollte ich sogar $99$ schreiben können. Wenn auf der Linie noch Platz ist (und Mama mich zu den Nachbarn gehen, die Straße überqueren und auf der Linie weitergehen lässt), müsste ich immer noch einen Schritt weitergehen und $1$ hinzufügen können, oder?\flqq{}\\
\frqq{} Das stimmt\flqq{}, antwortet Emilie, \frqq{}man kann immer einen Schritt weitergehen, also $1$ addieren. Nach $99$ kommt $100$, nach $999$ kommt $1000$, nach $9999$ kommt $10000$.\flqq{}\\
\frqq{}Also\flqq{}, fragt Ada, \frqq{}wer ist die größte Zahl? Denn wenn man immer $1$ addieren kann, bedeutet das, dass es immer eine Zahl gibt, die größer ist, und noch eine, die größer ist als die größte der größten Zahlen. Das hört nie auf!\flqq{}\\ 
Ada wurde fast schwindelig. \\
\frqq{}Das stimmt\flqq{}, erklärt Emilie, \frqq{}du hast Recht. Es hört nie auf. Wir sagen, dass es unendlich viele Zahlen gibt, und schreiben $\infty$.\flqq{}\\
\frqq{}Wie eine liegende Acht?\flqq{}\\
\frqq{}Ja, wie eine liegende Acht. Aber Vorsicht, die Unendlichkeit ist keine Zahl. Wie du schon vorher gesagt hast, kann man immer eine größere Zahl finden, und eine noch größere, und noch eine, die größer ist als die größte der größten, ohne jemals aufzuhören. Das meinen wir, wenn wir $\infty$ schreiben.\flqq{}\\
Ada schaut auf das neue Symbol. Sie sieht darin etwas wie einen Weg, der kein Ende hat. Sie glaubt, dass sie im Land der Zahlen sehr weit reisen kann.

%\begin{figure}[!hb]
%    \centering
%    \vspace{4cm}
%    \caption{}
%    \label{fig:infinity}
%\end{figure}