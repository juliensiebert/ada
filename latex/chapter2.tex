Ada continue de jouer aux additions et aux soustractions. 
Émilie lui a laissé son carnet et son stylo, puis elle est partie cueillir des fleurs dans le jardin. 
Après quelques parties, Ada a un problème. 
Elle n’a plus assez de chiffres ! 
Elle a noté $4 + 5 + 2 - 3$ dans le carnet, s’est placée sur le $4$, a avancé de cinq pas, et la voilà coincée. 
Elle est maintenant sur le $9$, le dernier chiffre de la ligne. 
Ada doit encore avancer de deux pas puis faire trois pas dans l’autre sens.
Mais voilà, après $9$, Ada n’a rien dessiné.
Elle pourrait avancer puis reculer sur la ligne sans chiffres ou poser ses pieds. Mais comment saura-t-elle si elle arrive au bon endroit ? 
Elle sait qu’il existe des choses après $9$. 
Elle a déjà entendu parler des nombres, comme vingt et un, soixante-trois, trente-six, ou encore quinze. 
Seulement, Ada ne sait pas comment les dessiner, ni même dans quel ordre.
Qui vient avant ? vingt et un ou quinze ? soixante-trois ou trente-six ?
La voilà bien préoccupée. Que faire ? Demander à sa tante Émilie ? Non, Émilie est certainement très occupée, et puis, Ada sent qu’elle peut y arriver toute seule. Alors Ada réfléchit. Elle se dit qu’elle pourrait inventer de nouveaux chiffres. 
Il n’y a qu'à dessiner de nouveaux symboles et leur trouver des nouveaux noms. Plus facile à dire qu'à faire. Après $4$ nouveaux chiffres, \qq{ga}, \qq{bu}, \qq{zeu}, et \qq{mo}, Ada est à court d'idées. 
Ce n’est pas si facile. Il faut trouver des nouveaux noms et puis imaginer de nouveaux symboles à dessiner (d’ailleurs qui a inventé les chiffres de $0$ à $9$ ? se demande Ada). Et puis, c’est difficile de retenir des choses pareilles. Déjà retenir les noms des chiffres de $0$ à $9$ lui avait donné du fil à retordre, alors imaginez si elle devait retenir un nouveau nom et un nouveau symbole à chaque nouveau nombre.
C’est qu’il peut y en avoir plein des nombres ! 
Alors, non, Ada décide qu’inventer des nouveaux chiffres ce n’est peut-être pas une si bonne idée après tout. 
Que faire ? Ada, toujours debout sur le $9$, doit encore avancer de deux pas et reculer de trois. Ada réfléchit de nouveau. Peut-être qu’elle peut réutiliser les chiffres ? 
Après $9$, on peut écrire de nouveau $0$, $1$, $2$, etc. Comme ça, pas besoin d’apprendre de nouveaux symboles. C’est bien, se dit Ada, mais le problème c’est qu’on peut se perdre. 
Comment saurais-je à quelle distance de la maison je me trouve? Si je me trouve sur un $5$, suis-je sur le premier $5$ - celui juste à côté de la maison, suis-je sur le deuxième ? le troisième ? Comment faire pour se repérer, se demande Ada.
Ada réfléchit. Et si elle utilisait de la couleur ? 
Une couleur pour la première série de chiffres de $0$ à $9$, par exemple le vert. Puis une autre couleur, disons le jaune, pour la deuxième série de chiffres, ensuite du rouge, puis du bleu. Hmm, est-ce que le bleu est plus grand que le rouge ? ou est-ce qu’il vaudrait mieux utiliser le bleu pour la troisième série et le rouge pour la quatrième? Et quelle couleur pour la cinquième ? Ada se dit que cette solution pose plus de problèmes qu’elle n’en résout.  
Les couleurs, ce n’est peut-être pas la bonne solution mais tout cela lui donne une autre idée. Ada va noter le nombre de fois qu’elle utilise les tous chiffres. Elle part de la maison: $0$, $1$, $2$, $3$, $4$, $5$, $6$, $7$, $8$, et $9$. Après $9$, Ada recommence la série de chiffres. Elle note $0$. Et pour ne pas oublier qu’elle vient d’utiliser toute la série de chiffres une première fois, elle note un $1$ à sa gauche. Ada obtient alors $1~0$ (\qq{un-zéro}). Et elle continue, après elle note $1~1$ (\qq{un-un}), $1~2$ (\qq{un-deux}), $1~3$ (\qq{un-trois}), $1~4$, $1~5$, $1~6$, $1~7$, $1~8$, et $1~9$. 
Arrivée là, Ada prolonge la ligne des chiffres sur le chemin du jardin. 
Puisqu’elle a utilisé toute la série de chiffres une deuxième fois, elle va maintenant noter $2~0$ (\qq{deux-zéro}) et continuer : $2~1$ (\qq{deux-un}), $2~2$ (\qq{deux-deux}), $2~3$, $2~4$, $2~5$. 
Ada s'arrête un instant pour regarder sa solution. Cela lui plaît. Tout d’abord, on est jamais perdu, pense-t-elle (on sait toujours si on est loin ou proche de la maison), et puis, on peut dessiner les chiffres dans la couleur que l’on veut ! 
Émilie, qui revient du jardin à ce moment-là, lui dit :\\
\guillemotleft Tiens, que fais-tu sur le vingt-cinq ?. \\
\mdash le vingt-cinq, demande Ada, le \qq{deux-cinq} s’appelle en fait vingt-cinq ?\\
\mdash Eh oui, répond Émilie, tous les nombres ont des noms.\\
\mdash Les nombres ? demande Ada.\\
\mdash Oui, dit Émilie, c’est comme ça qu'on les appelle. Les chiffres sont les symboles : $0$, $1$, $2$, $3$, $4$, $5$, $6$, $7$, $8$, et $9$. Il n’y en a que dix. Avec les chiffres on peut écrire des nombres : comme, par exemple, vingt-cinq avec un deux et un cinq : $25$.\\
\mdash Et comment il s’appelle celui avec un un et un quatre ? demande Ada.\\
\mdash Il s'appelle quatorze.\\
\mdash Et lui là ? le bizarre avec un 1 et un 7 ?\\
\mdash Lui, c’est dix-sept. \guillemotright\\ 
Émilie apprend alors à Ada, le nom des nombres: \qq{dix} ($10$), \qq{onze} ($11$), \qq{douze} ($12$), etc. jusqu’à $25$, là où Ada s’était arrêtée. 
Ada se demande bien qui décide du nom des nombres. Émilie lui explique que les chiffres ont beaucoup voyagé: Chine, Inde, Moyen Orient, Asie Centrale, Afrique du nord, Europe. 
Les chiffres tels qu’on les connaît sont utilisés partout dans le monde.
Ada pense que c'est bien pratique. Elle imagine un monde où les chiffres changent de forme lorsque l‘on passe d'un pays à l'autre. Comme ce serait compliqué !


%\begin{figure}
%    \centering
%    \begin{tikzpicture}[thick,scale=0.6, every node/.style={scale=0.6}, every arrow/.style={scale=0.6}]
%    % Grid
\draw[lightgray!20] (0,0) grid (12,12);

% Puzzle
\draw[line width=3pt, draw=black!75] (5,12) -- (0,12) -- (0,0) -- (6,0) -- (6,4) -- (7,4) -- (7,5) -- (9,5) -- (9,7) -- (10,7)
(7,0) -- (12,0) -- (12,12) -- (6,12)  -- (6,9) -- (3,9)
(1,0) -- (1,4)
(1,6) -- (1,5) -- (2,5) -- (2,2)
(2,3) -- (3,3) -- (3,1)
(2,1) -- (5,1) -- (5,2) -- (4,2)

(6,3) -- (5,3)
(2,5) -- (3,5) -- (3,4) -- (4,4) -- (4,3)
(4,4) -- (5,4) -- (5,5) -- (6,5) -- (6,6) -- (7,6) -- (7,7) -- (8,7) -- (8,8) -- (12,8)
(8,6) -- (9,6)
(6,1) -- (9,1) -- (9,4) -- (8,4) -- (8,3) -- (7,3) -- (7,2) -- (8,2)
(10,0) -- (10,1)
(11,1) -- (12,1)
(9,2) -- (11,2)
(10,3) -- (12,3)
(11,3) -- (11,7)
(9,4) -- (10,4) -- (10, 6)
(0,7) -- (2,7) -- (2,6) -- (4,6) -- (4,5)
(5,5) -- (5,7) -- (3,7)
(4,7) -- (4,8)
(0,8) -- (1,8)
(2,7) -- (2,9) -- (1,9)
(3,8) -- (3,10) -- (1,10) -- (1,11)
(2,12) -- (2,11)
(3, 11) -- (3,10) -- (5,10) -- (5,11)
(4,12) -- (4,11)
(5,9) -- (5,8)
(6,7) -- (6,8) -- (8,8) -- (8,9)
(6,9) -- (7,9)
(7,11) -- (7,10) -- (9,10) -- (9,9) -- (10,9) -- (10,8)
(8,11) -- (10,11) -- (10,10) -- (11,10) -- (11,9)
(10,12) -- (10,11)
(12,11) -- (11,11);

% Start and End Points
\draw[-latex,line width=3pt,red] (6.5,-1) -- (6.5,0);
\draw[-latex,line width=3pt,red] (5.5,12) -- (5.5,13);


% numbers all around (TODO Foreach loop)

\edef\n{1}
\foreach \x in {5.5,5.0,...,-0.5} {
    \node at (\x,-0.5) {\n};
    \pgfmathparse{int(Mod(\n+1,10))}
    \xdef\n{\pgfmathresult}
}

\foreach \y in {0.0,0.5,...,12.5} {
    \node at (-0.5,\y) {\n};
    \pgfmathparse{int(Mod(\n+1,10))}
    \xdef\n{\pgfmathresult}
}

\foreach \x in {0.0,0.5,...,4.5} {
    \node at (\x,12.5) {\n};
    \pgfmathparse{int(Mod(\n+1,10))}
    \xdef\n{\pgfmathresult}
}

\edef\n{1}
\foreach \x in {7.5,8.0,...,12.5} {
    \node at (\x,-0.5) {\n};
    \pgfmathparse{int(Mod(\n+1,10))}
    \xdef\n{\pgfmathresult}
}

\foreach \y in {0.0,0.5,...,12.5} {
    \node at (12.5,\y) {\n};
    \pgfmathparse{int(Mod(\n+1,10))}
    \xdef\n{\pgfmathresult}
}

\foreach \x in {12.0,11.5,...,6.5} {
    \node at (\x,12.5) {\n};
    \pgfmathparse{int(Mod(\n+1,10))}
    \xdef\n{\pgfmathresult}
}
%    \end{tikzpicture}
%    \caption{Ada est perdue au milieux des chiffres. Aide la à retrouver sa maison.}
%    \label{fig:maze}
%\end{figure}