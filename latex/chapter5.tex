Charles, le petit frère d'Ada est venu rejoindre sa soeur et Émilie pour le goûter. Après avoir mangé quelques gâteaux, Ada demande à Émilie : \\
\guillemotleft Dis, tata, est-ce que tu sais sauter à pieds joints? Comme ça regarde! \guillemotright\\
Ada se place sur le $0$, fait un premier saut et arrive sur le $2$. Un deuxième et la voilà sur le $4$. Encore un et Ada atterrit sur le $6$.\\
\guillemotleft Regarde, Émilie, à chaque saut j’avance de deux ! Allez à toi, maintenant ! \guillemotright\\
Émilie se place elle aussi sur le $0$, fait un premier saut et arrive sur le $3$. Encore un et la voilà sur le $6$ elle aussi. \\
\guillemotleft À moi ! s'écrie Charles. \guillemotright\\
Et lui aussi se met à sauter. Depuis le $0$, il atterrit sur le $1$. Puis sur le $2$, et le $3$, etc. jusqu'à arriver, après six sauts, avec Émilie et Ada sur le $6$.
Ada se demande alors où elle atterrirait si elle faisait 10 sauts d'affilée. 
Elle recommence depuis le début ($0$). Après trois sauts, la revoilà sur le $6$. 
Elle fait un quatrième saut et arrive sur le $8$. 
Après un cinquième la voilà sur le $10$. Elle continue ainsi en comptant le nombre de sauts et arrive, après dix sauts, sur le $20$. \\
\guillemotleft C’est à toi maintenant, dit Ada à Émilie. \guillemotright\\
Émilie se place sur le $0$ et fait un saut (elle atterrit sur le $3$), puis un deuxième (la voilà sur le $6$), et un troisième (sur le $9$), etc. Après le dixième saut la voilà bien plus loin qu’Ada, au nombre $30$.
Ada est étonnée, elle et Émilie ont fait chacune dix sauts. Mais comme les sauts d’Émilie sont plus grands que ceux d’Ada, Émilie est arrivée bien plus loin qu’elle.\\ 
\guillemotleft Et maintenant à toi Charles ! dit Ada à son frère. \guillemotright\\
Et Charles se met à sauter dix fois de suite. En partant du $0$, il arrive finalement au $10$.
Ada veut se souvenir de tous ces nombres. Elle prend son carnet et note :
\begin{description}[leftmargin=0.5cm]
    \item $2+2+2+2+2+2+2+2+2+2=20$ pour elle,
    \item $3+3+3+3+3+3+3+3+3+3=30$ pour Émilie, et
    \item $1+1+1+1+1+1+1+1+1+1=10$ pour Charles.
\end{description}
Comme c’est long et compliqué, pense Ada. 
Imaginez, si on devait faire 50 sauts ou encore 100, il n’y aurait pas la place de tout écrire! 
Il faudrait un moyen plus rapide d’écrire que l’on a fait 10 fois la même chose. On n’aurait alors besoin que de deux nombres: par exemple, le nombre de sauts ($10$), et la taille de chaque saut ($2$). Ce serait quand même plus pratique. 
Ada montre alors son carnet à Émilie et lui demande s'il n’existerait pas une manière plus simple d’écrire que $2+2+2+2+2+2+2+2+2+2=20$. Émilie lui présente alors un nouveau symbole: une petite croix qui s’appelle \guillemotleft fois \guillemotright et se dessine comme cela: $\times$. Émilie lui montre que que $2+2+2+2+2+2+2+2+2+2=20$ peut s’écrire $10 \times 2 = 20$ et qu’on dit dix fois deux.
Ada regarde ce nouveau venu puis écrit $10 \times 3 = 30$ dans son carnet. 
\\
\guillemotleft C’est quand même plus pratique que d’écrire $3+3+3+3+3+3+3+3+3+3=30$ dit-elle, et c’est exactement ce dont j’avais besoin. Et pour toi, Charles, on écrit $10 \times 1 = 10$. \guillemotright\\
Mais quelque chose préoccupe Ada. 
\guillemotleft Dis Émilie, demande-t-elle, tu m’as dit que quand on écrit $+$, on fait une addition, et quand on écrit $-$, on fait une soustraction. On fait quoi quand on écrit $\times$ ?\\
\mdash Cela s’appelle une multiplication, répond Émilie, on dit qu’on multiplie deux nombres entre eux.\\
\mdash Alors si je fait $50$ sauts j’arriverais à quel nombre ?”\\
\mdash Au cent, car $50 \times 2 = 100$.\\
\mdash Et si j’en fait $100$ ?\\
\mdash Au deux cents, car $100 \times 2 = 200$. \guillemotright\\
Ada imagine qu'elle serait très fatiguée à force de sauter autant.
