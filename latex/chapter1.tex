À ce moment, la tante d'Ada, Émilie, sort de la maison. \\
\guillemotleft Bonjour Ada, dit Émilie, comment vas-tu ?\\
\mdash Emilie !  crie Ada. \guillemotright \\
Ada aime beaucoup Émilie. C’est une de ses tantes préférées. Émilie est trop forte, elle a déjà beaucoup voyagé et parle trois langues différentes. En plus, le travail d’Émilie, c’est de fabriquer des fusées.\\
\guillemotleft Regarde Émilie, je joue aux chiffres ! explique Ada. J’ai dessiné tous les chiffres de 0 à 9. Et quand j’avance d’un pas, je passe au suivant. Et quand je recule d’un pas, je reviens au précédent. \guillemotright \\
Émilie explique alors à Ada, que quand elle avance d’un pas, on dit qu’elle ajoute un au chiffre sur lequel elle était, et que l’on le dessine $+1$. Lorsqu’elle recule d’un pas, elle soustrait un au chiffre sur lequel elle était, et que l’on le dessine $-1$. Ada a maintenant deux nouveaux symboles qu’elle peut dessiner : le plus ($+$) pour l’addition et le moins ($-$) pour la soustraction. Ada les trouve rigolos ces deux nouveaux. 
Ada et Émilie décident de jouer ensemble. En partant du $0$, Ada avance de trois pas ($+3$) puis recule de un pas ($-1$). La voilà arrivée sur le $2$. Émilie sort un carnet et un stylo et montre à Ada comment dessiner son trajet : $0 + 3 - 1 $. Et comme Ada est maintenant sur le $2$, Émilie lui explique qu’il y a un symbole nommé égal ($=$) que l’on utilise pour dire qu’Ada est arrivée sur le $2$ : $0 + 3 - 1 = 2$. Ada pense que cela ressemble à une formule magique. Elle a très envie de recommencer. \\
\guillemotleft Allez tata, maintenant tu écris les chiffres et les drôles de symboles dans ton carnet magique et moi je me déplace sur la ligne \guillemotright. \\
Émilie montre son carnet à Ada où est écrit : $4 + 3 - 2$. Ada réfléchit. \\
\guillemotleft Il faut commencer sur le 4, avancer de 3 pas, et reculer de 2, c’est ça ? \\
\mdash C’est ça ! répond Émilie. \guillemotright \\
Alors Ada se met en route. En partant du $4$, elle avance de trois pas et arrive sur le $7$. Puis elle se retourne et fait deux pas dans l’autre sens. Ada est maintenant arrivée sur le $5$. Émilie écrit alors $4 + 3 - 2 = 5 $.\\
\guillemotleft À moi, maintenant ! s’écrit Ada. \guillemotright \\
Émile lui donne alors carnet et stylo, puis Ada écrit : $1 + 7 - 3 $. Émilie regarde la feuille, se lève et marche vers le $5$.\\
\guillemotleft Voilà ! dit Émilie et elle note $1 + 7 - 3 = 5$ dans le carnet.\\
\mdash Quoi ? demande Ada. Tu triches ! il faut partir du $1$ avancer de $7$ pas et reculer de $3$ \guillemotright.\\
Et Ada lui montre. Elle se place sur le $1$, avance de sept pas (elle arrive alors au $8$) puis recule de trois pas, pour arriver exactement là où est Émilie, sur le chiffre $5$. Émilie explique alors à Ada que grâce au carnet et au stylo, on n’a pas toujours besoin d’avancer et de reculer. On peut calculer le résultat des opérations (c’est comme ça qu’Émilie appelle les formules magiques avec les chiffres, les $+$ et les $-$) et connaître le résultat (ce qui vient derrière le $=$) sans même bouger. Ada est sceptique. Elle demande à Émilie un autre exemple. Émile se place sur le $0$ et note $2 + 4 - 5 + 2 - 3$ dans le carnet. Ada se place sur le $2$, avance de quatre pas, puis en fait cinq dans l’autre sens, se retourne de nouveau pour avancer de deux pas et recule de trois pas. Ouf ! Ada en aurait presque la tête qui tourne. Émilie, elle, n’a pas bougé. Ada regarde alors sous ses pieds, et voit le $0$. Émile lui sourit et note $2 + 4 - 5 + 2 - 3 = 0$.\\
\guillemotleft Tu vois, dit-elle, je savais qu’on allait se retrouver toutes les deux ici sur le 0, avant même que tu ne te mettes en route.\\
\mdash C’est vrai, dit Ada, c’est pratique pour les formules très longues qui donnent le tournis. Mais moi, je trouve ça plus amusant de voyager pour de vrai ! \guillemotright.

%\begin{figure}[!h]
%    \centering
%    \begin{tikzpicture}[thick]
%    \node[rotate=80,scale=2.5] at (2,2) {\contour{black}{\textcolor{white}{$3+4=7$}}};
\node[rotate=15,scale=2] at (-5,2) {\contour{black}{\textcolor{white}{$8-3=5$}}};
\node[rotate=10,scale=3] at (-3,1) {\contour{black}{\textcolor{white}{$6-1-2=3$}}};
\node[rotate=170,scale=3] at (-0.5,-1) {\contour{black}{\textcolor{white}{$4-5+1=0$}}};
\node[rotate=-10,scale=2] at (-3,-2) {\contour{black}{\textcolor{white}{$9-1=8$}}};
%    \end{tikzpicture}
%    \caption{Colorie les formules magiques d'Ada et Émilie}
%    \label{fig:}
%\end{figure}