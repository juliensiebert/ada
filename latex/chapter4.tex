Ada se sent fière, elle a trouvé comment écrire les nombres plus grand que $9$ et elle a aussi découvert une infinité de nombres. 
Mais Ada est aussi un peu triste, car dans sa cour elle ne peut jouer qu’avec quelques nombres, et en plus ils sont petits : la plupart n’ont que deux chiffres !
Voyant cela, Émilie lui demande si elle veut encore jouer aux additions et aux soustractions. 
Mais cela n’amuse plus trop Ada. Elle à l’impression qu’il ne reste plus rien à découvrir. 
Émilie propose d’aller chercher un goûter. Mais avant de partir, elle note dans le carnet une nouvelle formule : $3 - 5 = ~$ et la tend à Ada. 
Une si petite formule se dit Ada, ce ne doit pas être bien compliqué. 
Je vais la résoudre (c’est comme ça que dit Émilie) avant qu’Émilie ne revienne. 
Ada se place sur le $3$ et commence à reculer : un pas, deux pas, trois pas et... 
Oh non, voilà que ça recommence ! Ada est maintenant sur le $0$, devant la porte de la maison, au début de la ligne. 
Il lui reste deux pas à faire pour finir sa formule mais la ligne s’arrête. 
Ada prend alors ses craies et continue la ligne derrière la maison jusqu’au garage. 
Elle voit bien qu’elle doit pouvoir reculer de deux pas afin de finir sa formule. 
Il y a assez de place pour faire encore au moins vingt pas, pense Ada. 
Le problème c’est qu’il n’y a pas de nombre de ce côté-ci de la ligne. 

Ada réfléchit. 
Elle se dit qu’elle pourrait réutiliser les nombres qu’elle connaît déjà. En partant de devant la maison ($0$), elle recule de un pas et note $1$, elle recule de deux pas et note $2$. 
Le problème, se dit Ada, c’est qu’on ne peut pas faire la différence entre le $1$ de derrière la maison et le $1$ qui est devant, dans la cour. 
Elle se dit qu’elle pourrait utiliser de la couleur. 
Vert pour devant et Orange pour derrière. 
La couleur c’est bien, mais Ada trouve un autre problème. 
Quand elle recule, c'est-à-dire quand elle enlève un, les nombres deviennent de plus en plus grands. Et quand elle avance, c’est-à-dire quand elle ajoute un, les nombres deviennent de plus en plus petits.
C’est l’inverse d’avant ! On ne peut pas avoir des règles pour un côté de la maison et d’autres règles pour l’autre côté. 
On va finir par tout mélanger, se dit Ada. 
Elle réfléchit encore, regarde son carnet avec toutes ses formules, et pense : quand c’est écrit $-1$, cela veut dire que j’enlève un et que je recule d’un pas. Quand c’est écrit $-2$, cela veut dire que j’enlève deux et que je recule de deux pas. Quand c’est écrit $-3$, cela veut dire que j’enlève trois, et que je recule de trois pas. 
Ada se place de nouveau devant la maison, sur le zéro. Elle recule d’un pas et note par terre $-1$ (j’ai reculé d’un pas depuis zéro). Elle recule encore d’un pas et note alors $-2$ (j’ai reculé de deux pas depuis zéro). Puis, elle continue à énumérer les pas et écrit par terre: $-3$, $-4$, $-5$, etc. 
Jusqu'à arriver au garage où Ada inscrit $-21$. 
À ce moment, Émilie revient avec le goûter.\\
\guillemotleft Tiens Ada, tu as découvert les nombres négatifs ! s’étonne sa tante. \guillemotright\\
Ada est fière d’elle. Émilie lui explique que les nombres négatifs ont eux aussi beaucoup voyagé et qu’ils ont longtemps été considérés comme des nombres bizarres. Ada se dit qu’ils n’ont rien de bizarre, et puis maintenant, grâce à eux, elle peut finir la formule d’Émilie. Ada se place sur le $3$ et recule de cinq pas. La voilà arrivée sur le $-2$.\\
\guillemotleft Voilà, dit Ada, $3 - 5 = -2$. Et maintenant, est-ce qu’on peut goûter ? \guillemotright

%\begin{figure}[!h]
%    \centering
%    \begin{tikzpicture}[thick,scale=0.6, every node/.style={scale=0.6}, every arrow/.style={scale=0.6}]
%    
\draw (-8,-3) -- (8,-3) -- (8,10) -- (-8,10) -- (-8,-3);
\draw (-1,0) -- (1,0) -- (1,2) -- (0,3) -- (-1,2) -- (-1,0)
(-1,2) -- (1,2);
\draw (0,-1) -- (7.5,-1);
\draw[dashed] (0,-1) -- (-2,-1);
\node at (0,-0.5) {0};
\node at (1,-0.5) {1};
\node at (2,-0.5) {2};
\node at (3,-0.5) {3};
\node at (4,-0.5) {4};
\node at (5,-0.5) {5};
\node at (6,-0.5) {6};
\node at (7,-0.5) {7};
%    \end{tikzpicture}
%    \caption{Comme Ada, continue la ligne de l'autre côté de la maison. Complete le dessin comme tu l'entends}
%    \label{fig:}
%\end{figure}