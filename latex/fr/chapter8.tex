Émile est maintenant de retour. Malheureusement sans bonbons. C’est dommage, pense Ada, mais peu importe, Ada doit d’abord montrer des choses à Émilie. Pour les bonbons, on verra plus tard.\\
\guillemotleft Regarde Émilie, dit Ada, Charles fait des demi-pas. Quand je fais un pas, il doit en faire deux.\\
\mdash Ah oui? Et combien vaut un pas d’Émilie en pas d’Ada, demande sa tante?\\
\mdash Je ne sais pas, mais on a qu’à essayer, répond Ada. J’ai une idée. On se met l’une à côté de l’autre, on marche ensemble et on regarde combien de pas on doit faire pour arriver au même endroit.\guillemotright\\
Ada et Émilie s’installent toutes les deux devant la maison, sur le $0$. Ada fait un pas et arrive sur le $1$. Émilie fait également un pas mais ses pas sont plus grands et elle est coincée entre le $1$ et le $2$.
Ada fait un deuxième pas et passe devant Émilie. Émilie fait un deuxième pas et arrive sur le $3$.  Ada doit faire un troisième pas pour rejoindre Émilie.
Ada note alors dans son carnet $3$ Ada $= 2$ Émilie. Trois pas d’Ada valent deux pas d’Émilie. Cela parait bien compliqué se dit Ada. J’aimerai mieux savoir combien un de mes pas fait en pas d’Émile. Si trois de mes pas font deux pas d’Émilie, alors en coupant en trois, j'obtiens un pas d’Ada égal deux pas d’Émilie coupés en trois. Ada note $1$ Ada $= 2/3$ Émilie.
Émilie regarde le carnet et s’exclame : \\
\guillemotleft Oh les belles équations!\\
\mdash Les équa-quoi? s’étonne Ada.\\
\mdash Les équations, reprend Émilie. Ce sont les formules que tu viens de noter. C’est comme ça qu'on les appelle. Elles sont comme des recettes qui nous disent comment transformer des choses.\\
\mdash Je le savais, dit Ada, on peut mélanger les chiffres et les lettres.\\
\mdash Oui, du moment que le compte est bon.\\
\mdash Je me demande combien de pas tu dois faire si j’avance jusqu’au 12, dit Ada.\\
\mdash Et bien, on pourrait essayer, répond Émilie. Je me place sur le $0$. J’avance de un, deux pas et je suis sur le $3$. Après trois et quatre pas, me voici sur le $6$. Cinq et six pas, sur le $9$. Sept et huit pas. Et voilà! J’arrive sur le $12$. Huit pas, voilà ta réponse Ada.\\
\mdash Hé! ne me laissez pas tout seul! crie soudain Charles depuis la maison.\\
\mdash Tu n’es pas tout seul, répond Ada, et en plus, tu n'es même pas loin.\\
\mdash Si je suis très loin, au moins cent cinquante huit!\\
\mdash N'importe quoi, dit Ada, nous sommes au $12$ et tu es sur le $0$. Puisque tu fais des demi-pas, tu dois faire le deux fois plus de pas que moi. C'est-à-dire deux paquets de 12 et ça fait..., Ada calcule dans sa tête, vingt-quatre, pas vrai Émilie?\\
\mdash Oui c'est ça, allez viens Charles, tu vois ce n’est pas si loin que ça.\guillemotright\\
Charles court les rejoindre.\\ 
\guillemotleft Et maintenant, dit Émilie, je dois aller chercher de la salade dans le jardin. Qui vient avec moi?\\
\mdash Moi! s'écrie Charles!\\
\mdash Et toi, tu viens Ada?\\
\mdash Je ne sais pas, les salades sont au fond du jardin et ça, ça me parait loin. Ce qui serait bien, continue Ada, c’est d'avoir une méthode qui permette de savoir combien de pas je dois faire pour te rejoindre. Le mieux, ce serait d'avoir la réponse tout de suite en regardant, sans calculer.\\
\mdash Je peux te dessiner les équations, propose Émilie.\\
\mdash Les dessiner? reprend Ada, mais je les ai déjà écrites dans mon carnet!\\
\mdash Non, regarde.\guillemotright\\
Emilie dessine deux lignes dans le carnet : une première horizontale pour ses pas à elle et une deuxième verticale pour les pas d’Ada. Émilie note alors $0$ à l’endroit ou les lignes se croisent, puis elle ajoute des chiffres de $1$ à $9$ sur chacune. Elle montre à Ada comment dessiner l'équation. Quand Émilie fait deux pas, Ada doit en faire trois. Émilie pose le stylo sur le numéro $2$ de la ligne horizontale et monte jusqu'au niveau du numéro $3$ de la ligne verticale. Là, elle marque un premier point. Quand Émilie fait quatre pas, Ada doit en faire six. De la même manière Émilie marque un deuxième point. Quand Émilie fait six pas, Ada doit en faire neuf. Finalement Émilie marque un troisième point sur la feuille. Ensuite, elle trace une ligne reliant tous les points.\\
\guillemotleft Tu vois, dit Émilie, cette ligne que je viens de tracer c’est elle qui représente mon équation. Si tu veux savoir combien de pas tu dois faire pour me rejoindre, il te suffit de suivre cette ligne.\\
\mdash Suivre la ligne?\\
\mdash Oui, regarde. Si je fais huit pas, la ligne me dit que tu dois en faire douze.\\
\mdash Ça je le savais déjà! répond Ada.\\
\mdash Si je fais dix pas, tu dois en faire quinze.\\ 
\mdash Et si tu n’en fais que cinq? demande Ada, elle nous dit quoi la ligne?\\
\mdash Elle nous dit que tu dois en faire sept et demi.\guillemotright\\
Ada regarde le dessin dans son carnet. Cela lui plaît bien, après les formules magiques et les recettes, voici une carte au trésor.
