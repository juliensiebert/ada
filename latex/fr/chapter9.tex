Pendant qu'Émilie cueille de la salade, Ada et Charles jouent ensemble à "Ada a dit". Le but est simple : Ada donne des ordres à Charles et celui-ci ne doit les exécuter que si Ada à prononcer la formule "Ada a dit".\\
\guillemotleft Ada a dit, recule de trois pas!\guillemotright\\
Charles recule effectivement de trois pas.\\
\guillemotleft Bravo! Ada a dit, saute à cloche-pied!\guillemotright\\
Charles se met sur un pied et commence à sauter.\\
\guillemotleft Maintenant, avance de cinq pas en courant!\guillemotright\\
N'y tenant plus, Charles se met à courir vers Ada.\\
\guillemotleft Perdu! Je n'ai pas dit "Ada a dit".\\
\mdash Oh non! Allez c'est à moi! Charles a dit $3 \times (-2)$!\\
\mdash $3 \times (-2)$!? s'étonne Ada. Attend un peu... Ada réfléchit et dit : $-2$ ça veut dire qu'on recule de deux pas, donc $3 \times (-2)$ ça veut dire qu'on recule trois fois de deux pas.\guillemotright\\
Ada recule de six pas.\\
\guillemotleft Charles a dit $(-2) \times 3$!\guillemotright\\
Ada recule de nouveau de six pas.\\
\guillemotleft Hé tu triches! s'écrie Charles.\\
\mdash Comment-ça je triche? réplique Ada.\\
\mdash Ben oui! Avant c'était $3 \times (-2)$ et maintenant c'est $(-2) \times 3$, c'est pas pareil donc ça ne peut pas faire la même chose.\\
\mdash $(-2) \times 3$ c'est reculer de deux pas trois fois, c'est comme reculer trois fois de deux pas. $3 \times (-2) = (-2) \times 3$.\guillemotright\\
À ce moment Émilie arrive avec une salade dans chaque main.\\
\guillemotleft Ada a raison. La multiplication est commutative.\\
\mdash Commuta-quoi? demandent en choeur les frère et soeur.\\
\mdash Commutatif. Cela veux dire que l'on peux changer de place les nombres autour du signe $\times$ : $5 \times 4 = 4 \times 5$, $7 \times 3 = 3 \times 7$, ou encore $3 \times (-2) = (-2) \times 3$. Cela marche aussi pour l'addition : $1 +3 = 3 + 1$, $5 + 2 = 2 + 5$.\\
\mdash Et la soustraction et la division ? demande Ada.\\
\mdash Elles, elles ne sont pas commutatives. On ne peux pas déplacer les nombres comme on veut : $3 - 2$ n'est pas égal à $2 - 3$ et $6/3$ n'est pas égal à $3/6$. \guillemotright\\
Ada regarde Charles.\\
\guillemotleft Ça veut dire que j'ai gagné!\\
\mdash Non, encore un! Charles a dit $(-4) \times (-3)$!\guillemotright\\
Ada réfléchit. Elle n'a encore jamais multiplié deux nombres négatifs\\
\guillemotleft Quand on avance de deux pas, on écrit $2$ et quand on recule de deux pas on écrit $(-2)$. Si on avait $4 \times (-3)$, quatre paquets de $(-3)$, on reculerait quatre fois de trois pas.\guillemotright\\
Ada calcule. \\
\guillemotleft On reculerait de 12 pas. Et comme on a $(-4) \times (-3)$, on fait la même chose dans l'autre sens. On avance de 12 pas.\guillemotright\\
Ada demande à Émilie.\\
\guillemotleft Dis tata, est-ce que $(-4) \times (-3)$ ça fait $12$ ?\\
\mdash Bravo Ada, répond Émilie, quand on fait des multiplications avec des nombres positifs et négatifs on doit jongler avec les signes.\\
\mdash Jongler avec les signes? Ça a l'air difficile.\\
\mdash Non, tu viens de le faire. Multiplier deux nombres positifs ensemble ou deux nombres négatifs ensemble donne toujours un résultat positif. Multiplier un nombre positif avec un nombre négatif donne un résultat négatif.\guillemotright\\
Soudain depuis la maison, une voix appelle les enfants:\\
\guillemotleft Ada, Charles! Il est l'heure de rentrer!\\
\mdash On arrive! répondent à l'unisson les frère et soeur.\guillemotright\\