Ada reprend son jeu là où elle s’était arrêtée. 
Elle prend le carnet et lit $4 + 5 + 2 - 3$. 
Ada se place sur le $4$ et avance de cinq pas. 
Elle est maintenant sur le $9$. 
Elle avance de deux pas. La voilà sur le $11$. 
Ada se retourne et fait $3$ pas dans l’autre sens et s’arrête sur le $8$.
Enfin ! Ada peut maintenant écrire $4 + 5 + 2 - 3 = 8$ dans son carnet.
Elle se tourne vers Émilie et lui demande :\\
\guillemotleft Dis tata, c’est qui le plus grand nombre ? \guillemotright\\
Émilie la regarde en souriant.
\guillemotleft Pour l’instant le plus grand nombre que tu as dessiné sur ta ligne c’est vingt-cinq.\\
\mdash Non ! réplique Ada, tu sais bien, le plus grand de TOUS les nombres. Après 25, je peux écrire $2~6$ (Ada prononce alors \qq{deux-six} et Émilie lui indique son nom: \qq{vingt-six}), et $2~7$ (\qq{deux-sept}, \qq{vingt-sept}), $28$, $29$. Après, j’écris $30$, pour dire que j’ai utilisé trois fois tous les chiffres de $0$ à $9$. Et je continue: j’écris $31$, $32$, $33$ et ainsi de suite. Tu vois, si la ligne continue dans le jardin, chez les voisins, si elle traverse la rue, je devrais même pouvoir écrire $99$. S'il y a encore de la place sur la ligne (et si maman me laisse aller chez les voisins, traverser la rue et continuer sur la ligne), je dois encore pouvoir avancer d’un pas, ajouter $1$, pas vrai ?\\
\mdash C’est vrai, répond Émilie, on peut toujours avancer d’un pas, c’est-à-dire ajouter $1$. Après $99$ vient $100$, après $999$ vient $1000$, après $9999$ vient $10000$.\\
\mdash Alors, demande Ada, c’est qui le nombre le plus grand ? Parce que si on peut toujours ajouter $1$, cela veut dire qu’il y a toujours un nombre plus grand et encore un plus grand que le plus grand des plus grand. Ça ne finit jamais ! \guillemotright\\ 
Ada en aurait presque le vertige. \\
\guillemotleft C’est vrai, explique Émilie, tu as raison. Ça ne finit jamais. On dit qu’il y a une infinité de nombres et on écrit $\infty$.\\
\mdash Comme un huit couché?\\
\mdash Oui, comme un huit couché. Mais attention, l’infini n’est pas un nombre. Comme tu l’as dit avant, on peut toujours trouver un nombre plus grand, et un autre encore plus grand, et encore un plus grand que le plus grand des plus grands, sans jamais s'arrêter. C’est ce qu’on veut dire quand on écrit $\infty$. \guillemotright\\
Ada observe ce nouveau venu. En le regardant, elle voit comme un chemin qui ne finit pas. Elle se dit qu’elle pourrait voyager très loin au pays des nombres.

%\begin{figure}[!hb]
%    \centering
%    \vspace{4cm}
%    \caption{}
%    \label{fig:infinity}
%\end{figure}