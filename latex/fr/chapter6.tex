Ada a encore un peu faim.
Il n'y a plus de gâteau mais, heureusement, Émilie a rapporté des framboises du jardin.
Il y en a douze. Ada se demande comment les partager. 
C'est qu'Émilie et Charles aiment aussi les framboises. 
Ada réfléchit. Je vais commencer par en donner une pour Charles, une pour Émilie et une pour moi. 
Il en reste (Ada compte les framboises à haute voix) neuf. 
En voilà une deuxième pour Charles, une deuxième pour Émilie et une deuxième pour moi. Il n'en reste plus que six. 
Alors en voici une troisième pour Charles, Émilie et moi. Il n'en reste maintenant plus que trois. 
Pour finir Ada, donne une quatrième framboise à chacun. Et voilà les douze framboises partagées.\\
\guillemotleft Qui a encore faim? Qui veut des framboises ? demande Ada. Regardez, on en a tous quatre.\\
\mdash Bravo, dit Émilie, tu connais aussi la division !\\
\mdash La quoi ? s'étonne Ada.\\
\mdash La division, explique Émilie. C'est ce que tu viens de faire pour partager les framboises. La division, c'est la copine de la multiplication, de la soustraction et de la multiplication.\\
\mdash Ah bon ? répond Ada, la bouche pleine de fruits. Et on la dessine comment ?\\
\mdash On utilise souvent un trait pour montrer qu'on divise (on coupe) un nombre par un autre.\\ 
\mdash On coupe les nombres ! s'écrie Ada. \guillemotright\\
Ada imagine des nombres coupés en deux. Un $8$ ayant perdu sa moitié haute (\clipbox{0 -0.5 0 {0.4\totalheight}}{$8$}), un $4$ sans pied (\clipbox{0 {0.3\totalheight} 0 0}{$4$})! Bizarre.\\
\guillemotleft Et ça ne leur fait pas mal ?\\
\mdash Mais non, la rassure Émilie. Tiens regarde. \guillemotright\\
Émilie prend le carnet et note: $12/3=4$.\\
\guillemotleft Douze framboises partagées en trois égalent 4 framboises par personne. On peut aussi écrire comme cela. \guillemotright\\
Émilie tend le carnet à Ada et lui montre : 
$$\frac{12}{3}=4$$
Ada la trouve plutôt jolie la division. Elle se demande si ça marche pour tous les nombres et pas juste pour les framboises. 
Elle aimerait savoir combien font $7$ divisé par $3$. Comment faire ? Pour les framboises, c'était facile, il suffisait de les distribuer. Une pour Ada, une pour Charles, une pour Émilie, deux pour Ada, deux pour Charles, deux pour Émilie, etc. jusqu'à ce qu'il n'y ait plus de framboises. 
Mais, Ada n'a plus de framboises pour essayer. Ada réfléchit. Elle a une idée. Elle va utiliser ses craies de couleur et des cailloux.
Ada se lève et prend trois craies : une verte, une orange et une violette. Puis elle dessine trois ronds par terre : un vert, un orange et un violet. Ensuite, elle ramasse sept cailloux. Elle en met d'abord un dans le rond vert, un dans le orange et un dans le violet. Voilà, c'est comme pour les framboises, sauf que là, Ada distribue les cailloux. Ada continue. Il lui reste quatre cailloux. Elle en en met un dans le rond vert, un dans le orange et un dans le violet. Chaque rond à maintenant deux cailloux. Ada regard dans sa main. Il ne reste qu'un seul cailloux à distribuer. Plus assez pour en déposer un dans chaque rond de couleur ! Ça ne va pas aller.
\guillemotleft Émilie ? Ta division, elle ne fonctionne pas.\\
\mdash Ah bon ? s'étonne Émilie. \\
\mdash Oui, reprend Ada, je voulais diviser $7$ par $3$. J'ai trois paquets : un vert, un orange et un violet. J'ai mis deux cailloux dans chaque paquet. Mais il me reste encore un partager pour finir et je ne peux pas le couper en trois !\\
\mdash C'est normal, explique Émilie, tous les nombres ne se laissent pas toujours diviser entièrement. Quelquefois, il y a un reste. \\
\mdash Un reste ? \\
\mdash Oui un reste, c'est comme ça que ça s'appelle. Dans ton cas, $7$ divisé par $3$ égale $2$ et il reste $1$. On peut aussi dire que $7$ égale $3$ paquets de $2$ plus $1$. C'est pareil. Et on peut le noter comme cela : $7=3\times2+1$. \\
\mdash Hein ? s'exclame Ada. Elle est passée où ma division ? Tu triches encore, tata ! \\
\mdash Mais bien sûr que non, la multiplication et la division sont tellement copine que l'on peut changer l'une en l'autre. On dit que la division est l'inverse de la multiplication. \guillemotright\\
Ada la regarde stupéfaite. Puis après quelques temps elle dit : \\
\guillemotleft D'accord, alors s'il te plaît, est-ce que tu peux nous chercher six bonbons ? Deux chacun ? Parce que $6 = 3 \times 2$ et $6 / 3 = 2$, n'est-ce pas? \guillemotright 

